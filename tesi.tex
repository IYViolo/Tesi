
\documentclass[12pt]{article}
\usepackage[english]{babel}
\usepackage{amsfonts}
\usepackage{extsizes}
\usepackage[utf8]{inputenc}
\usepackage{amsthm}
\usepackage{ mathrsfs }
\usepackage{ amsmath }
\usepackage{ amssymb }
\usepackage{enumerate}
\usepackage{cite}
\usepackage{bbm}


\theoremstyle{definition}
\newtheorem{definition}{Definition}

\newtheorem{remark}{Remark}
\newtheorem{theorem}{Theorem}
\newtheorem{lemma}{Lemma}
\newtheorem{prop}{Proposition}

\DeclareMathOperator\supp{supp}
\DeclareMathOperator\Lip{Lip}
\DeclareMathOperator\rr{\mathbb{R}}
\DeclareMathOperator\nn{\mathbb{N}}
\DeclareMathOperator\wt{\widetilde}
\newcommand\addtag{\refstepcounter{equation}\tag{\theequation}}


\begin{document}





\section{Introduction}
An extension operator is a functional operator $E$ that allows to extend functions defined on a subset $\Omega$ of $\rr^n$ to the whole $\rr^n$, preserving some properties like regularity and summability. The first notable works on this topic are due to Whitney[...] and Hestenes[...] who treated the problem of extending functions from the space $C^m(\Omega)$ to the space $C^m(\rr^n)$, with $m\in\nn $ and where $\Omega$ is a closed set of $\rr^n$. They both showed that for every $m$ there exists an extension operator valid for $C^m(\Omega)$. The main difference is that while the Hestenes operator requires the boundary of $\Omega$ to be of class $C^m$, the Whitney operator works for any closed set $\Omega.$ Concerning the extension of Sobolev spaces, a first result was published by Calderon [...]. He showed that, given an open set with Lipschitz boundary, for every $l \in \nn$ exists a linear operator that extends function from $W^{l,p}(\Omega)$ to $W^{l,p}(\rr^n)$ continuously, for every $1<p<\infty$. A remarkable improvement was done by Stein \cite{stein}. For any open set $\Omega$ with Lipschitz boundary he constructed a bounded linear extension operator that extends every space $W^{l,p}(\Omega)$ with $1\le p\le \infty$ and $l \in \nn.$ A further way of extending Sobolev spaces was given by Burenkov \cite{burenkov}, similarly to the work of Calderon, he gave for every $l\in \nn$ an operator that extend continuously the space $W^{l,p}(\Omega)$ for any $1\le p\le \infty$. \\
Our main goal will be to study the problem of the extension of Sobolev-Morrey spaces.
\section{Notations and basic inequalities}

We will use the following standard notations for sets:\\
$\rr$ - the set of all real numbers,\\
$\nn$ - the set of all natural numbers,\\
$\nn_0$ - the set of all nonnegative integers,\\
$\nn_0^n=\underbrace{\nn_0 \times ....\times \nn_0}_{n}$, with $n \in \nn,$\\
$\rr^n=\underbrace{\rr \times ....\times \rr}_{n}$, with $n \in \nn,$\\
$B_r(x)$ - The ball of radius $r>0$ centered at a point $x\in \rr^n,$ \\ 
$\overline \Omega$ ($\Omega \subset \rr^n$) - the closure of $\Omega$. \\
We will sometimes denote a point $x \in \rr^n$ by $(\overline x ,x_n)$, where $\overline x \in \rr^{n-1}$ and $x_n \in \rr.$\\\\
For any $\alpha=(\alpha_1,...,\alpha_n) \in \nn_0^n$ we set:
\[ |\alpha|=\alpha_1+...+\alpha_n,\]
\[ \alpha!=\alpha_1!\cdots\alpha_n!.\]
Given $\alpha,\beta \in \nn_0^n$ we say that $\beta\le \alpha$ if $\beta_i\le\alpha_i$ for every $i \in \nn.$\\
Given $\alpha,\beta \in \nn_0^n$ such that $\beta \le \alpha$ we set 
\[ {\alpha \choose \beta} = \frac{\alpha!}{(\alpha-\beta)!\beta!}. \]
For $\alpha \in \nn_0^n$, $\alpha \neq0$ we will write\\
$D^\alpha f=\frac{\partial^{|\alpha|}f}{\partial^{\alpha_1}x_1...\partial^{\alpha_n}x_n},$ the ordinary derivative of order $\alpha$ of the function $f.$  In the case $\alpha=0$ we agree that $D^\alpha f=f.$ \\\\
For an arbitrary nonempty set $\Omega$ of $\rr^n$ we shall write:\\
$C(\Omega)$ - the space of continuous functions in $\Omega,$\\
$\Lip(\Omega)$ - the space of functions defined in $\Omega$ such that
\[ |f(x)-f(x)|\le M |x-y|\]
for every $x,y \in \Omega$. The best constant $M$ such that the previous inequality holds for every $x,y \in \Omega$ is called Lipschitz constant of $f$ and will be denoted by $\Lip f.$ $\Lip(\Omega)$ will be also called the space of Lipschitz functions in $\Omega.$ \\\\
 Lipschitz function in $\Omega.$ Moreover for a function $f \in \Lip(\Omega)$ we will denote by $\Lip f$ the Lipschitz constant of $f$.\\ \\
For an arbitrary measurable set $\Omega$ of $\rr^n$ we will denote by:\\
$L^p(\Omega)$ ($1\le p<\infty$) - the space of measurable functions $f$ on $\Omega$ such that 
\[ \|f\|_{L^p(\Omega)}=\left(\int_\Omega |f|^p\right)^\frac{1}{p}<\infty, \]
$L^\infty$ - the space of measurable functions $f$ on $\Omega$ such that
\[ \|f\|_{L^\infty(\Omega)}=\text{ess } \sup_{x \in \Omega} |f(x)|=\inf_{\omega: \  |\omega|=0} \sup_{x \in \Omega\setminus \omega} |f(x)|<\infty.\]\\
For an arbitrary open set $\Omega$ of $\rr^n$ we shall write:\\
$L^p_{loc}(\Omega) $ ($1\le p\le \infty$) - the set of measurable functions on $\Omega$ such that $f \in L^p(K)<\infty$  for every $K$ compact subset of $\Omega,$ 		\\
$C^l(\Omega)$ ($l \in \nn$) - the set of functions $f$ defined on $\Omega$ such that, for every $\alpha \in \nn_0^n$ with $|\alpha|=l$ and for every $x \in \Omega $, $D^\alpha f(x) $ exists and $D^\alpha f \in C(\Omega),$
$C^\infty(\Omega) = \bigcap_{l\in \nn} C^l(\Omega)$ - the set of the infinitely continuously differentiable functions in $\Omega$, \\ 
$C^l_c(\Omega)$ ($l \in \nn$) - the set of functions $f$ in $C^l(\Omega)$ with compact support,\\
$C^\infty_c(\Omega)$  - the set of functions $f$ in $C^\infty(\Omega)$ with compact support.\\\\
Let $\Omega$ be a measurable set in $\rr^n$ and $f$ a measurable function on $\Omega$, we will denote by
\[ \supp f= \text{ess } \supp f= \Omega \setminus \bigcup_{\substack{X \text{open} \\ f=0 \text{ a.e. in } X}} X.\]

Let $\Omega$ a measurable set in $\rr^n$ and $1\le p<\infty.$\\
\textbf{Minkowski's inequality.} If $f,g \in L^p(\Omega)$, then $f+g\in L^p(\Omega)$ and
\[ \|f+g\|_{L^p(\Omega)}\le \|f\|_{L^p(\Omega)}+\|g\|_{L^p(\Omega)}.\]
\textbf{Minkowski's integral inequality.} Let $A$ a measurable set in $\rr^n.$ Suppose that $f$ is a measurable function on $\Omega\times A$ and that $f(.,y) \in L^p(\Omega)$ for almost all $y\in A$, then
\[ \left\| \int_A |f(.,y)|^p dy \right \|_{L^p(\Omega)} \le \int_A \|f(.,y)\|_{L^p(\Omega)}dy\]
\textbf{Leibniz Rule} Let $f,g$ functions in $\rr^n$ differentiable up to order $l$. Then for every $\alpha \in \nn^n_0$ with $0<|\alpha|\le l$
\[ D^\alpha(fg)=\sum_{\beta \le \alpha} {\alpha \choose \beta} D^{\alpha-\beta}fD^\beta g.\]
\section{Preliminaries}
In this section we will recall some classical definitions and results that will be used along the exposition. We start with some basic theory about Sobolev spaces and weak derivatives, then proceed defining the notion of open set with Lipschitz (and $C^m$) boundary. We will conclude with the definition of Morrey spaces. 
\begin{definition}[Weak derivatives]
	Let $\Omega$ be an open set in $\rr^n$, $f \in L^1_{loc}(\Omega)$ and  $\alpha\in \nn^n_0$ with $\alpha \neq 0.$ A weak derivative of order $\alpha$ of $f$ is a function $g \in L^1_{loc}(\Omega)$ such that
	\[\int_\Omega f(x)D^\alpha\phi(x)dx=(-1)^{|\alpha|} \int_\Omega g(x)\phi(x)dx\]
	for every $\phi \in C_c^\infty(\Omega).$ In symbols we will write that $g=D^\alpha_wf.$
\end{definition}
The next proposition gives a characterization of the weak derivatives of a function.
\begin{prop}\label{weak2}
	Let $\Omega$ be an open set in $\rr^n$, $\alpha \in \nn^n_0$ with $\alpha \neq 0$ and $f,g \in L^1_{loc}(\Omega).$ The function $g$ is a weak derivative of $f$ of order $\alpha$ if and only if there exists a sequence $\{\psi_k\}_{k\in\nn}$ of real-valued functions of class $C^\infty(\Omega)$ such that 
	\begin{itemize}
		\item the sequence $\{\psi_k\}_{k\in\nn}$ converges to $f$ in $L^1_{loc}(\Omega)$ as $k \to \infty$,
		\item the sequence $\{D^\alpha\psi_k\}_{k\in \nn}$ converges to $g$ in $L^1_{loc}(\Omega)$ as $k \to \infty$.
	\end{itemize}
\end{prop}
It's important to remark that the existence of all the weak derivatives of some order of a function, implies the existence of all the weak derivatives of lower order. In particular we have the following result

\begin{prop}
	Let $\Omega$ be an open set in $\rr^n$, $n \ge2$ and $l \in \nn$ with $l\ge2$. Assume also that $f \in L^1_{loc}(\Omega)$ and that the weak derivative $D^\alpha_wf $ exists for every $\alpha\in \nn^n_0$ with $\alpha=l$. Then for every $\beta \in \nn^0_n$ such that $|\beta|<l$ and $\beta \neq 0$ the weak derivative $D^\beta_w f$ exists.
\end{prop}
The following two results shows that the weak derivatives behave similarly to the classical derivatives with respect to product and the composition of functions. 
\begin{prop}[Leibniz rule for weak derivatives]
	Let $\Omega$ be an open set in $\rr^n$, $l\in \nn$ and $\psi$ be a real-valued function of class $C^\infty(\Omega)$.  Assume that $f \in L^1_{loc}(\Omega)$ and that the weak derivative $D^\alpha_wf$ exists for every $|\alpha|\le l.$ Then also the weak derivative $D^\alpha_w(\psi f)$ exists and  
	\[ D^\alpha(\psi f)=\sum_{\beta \le \alpha} {\alpha \choose \beta} D^{\alpha-\beta}_wfD^\beta \psi.\]
\end{prop}
The next proposition related to the chain rule for weak derivatives contains also a useful bound (inequality \eqref{compbound}) that will be used in several proofs.
\begin{prop}[Chain rule for weak derivatives]\label{composition}
Let $l \in \nn$ and $\Omega$ be a domain in $\rr^n$. Suppose that $f \in L^1_{loc}(\Omega)$ admits all the weak derivatives up to order $l$ and that $g:\Omega'\rightarrow \Omega$ is a diffeomorphism of class $C^l$ with bounded derivatives $|D^\alpha g_k|\le M$ for all $1 \le |\alpha| \le l$. Then $f\circ g$ admits weak derivative up to order $l$. Moreover for every $1\le |\alpha|\le l$ we have to following bounds
\begin{equation} |D^\alpha (f\circ g)|\le C \sum_{1\le|\beta|\le |\alpha|} |D^\beta f(g)| \label{compbound}
\end{equation}
where $C$ depends only on $M$ and $l$.
\end{prop}
\begin{proof}
We prove the statement by induction on $l$. For $l=1$ we know from Proposition \ref{weak2} that exists a sequence of functions $\{f_k\}_k  \in C^\infty(\Omega)$ such that
\[ f_k \rightarrow f \qquad \text{in } L^1_{loc}(\Omega) \]
\[ \frac{\partial f_k }{\partial x_i}\rightarrow \frac{\partial f }{\partial x_i} \qquad \text{in } L^1_{loc}(\Omega),\]
where $\frac{\partial f }{\partial x_i}$ denote the weak derivatives of $f$ first order. Take $\phi \in C^\infty_c(\Omega')$ and integrate by parts
\[\int_{\Omega'} f_k(g(x))\frac{\partial \phi }{\partial x_i}(x)dx = - \int_{\Omega'} \left(\sum_{j=1}^n \frac{\partial f_k }{\partial x_j}(g(x))\frac{\partial g_j }{\partial x_i}(x)\right)\phi(x)dx.\]
Since $\phi(g^{-1}) \in C^l_c(\Omega)$ and the derivatives of $g$ and $g^{-1}$ are bounded, we can pass to the limit in the above equation
\[\int_{\Omega'} f(g(x))\frac{\partial \phi }{\partial x_i}(x)dx = - \int_{\Omega'} \left(\sum_{j=1}^n \frac{\partial f }{\partial x_j}(g(x))\frac{\partial g_j }{\partial x_i}(x)\right)\phi(x)dx.\]
Hence the case $l=1$ is proved. Now suppose that the statement is true for $l$. We prove the case $l+1$, so we suppose that $f$ admits weak derivatives up to order $l+1$ and that $g$ is of class $C^{l+1}$. From the case $l=1$ we know that $\frac{\partial (f \circ g) }{\partial x_i}$ exists and that
\[ \frac{\partial (f \circ g) }{\partial x_i}= \sum_{j=1}^n (\frac{\partial f }{\partial x_j} \circ g)\frac{\partial g_j }{\partial x_i}\] 
Since $\frac{\partial f }{\partial x_j}$ admits weak derivatives up to order $l$, by induction hypothesis the functions $\frac{\partial f }{\partial x_j} \circ g$ admit weak derivatives up to order $l$. Moreover $\frac{\partial g_j }{\partial x_i}$ is of class $C^l$, thus by the Leibniz rule the functions $(\frac{\partial f }{\partial x_j} \circ g)\frac{\partial g_j }{\partial x_i}$ admits weak derivatives of order $l$. In conclusion $\frac{\partial (f \circ g) }{\partial x_i}$ admits derivatives up to order $l$ and this conclude the proof of the case $l+1$. 

To prove the bounds we notice that the weak derivatives $D^\alpha (f \circ g)$ can be computed using the chain rule for usual derivatives. Such formula can be found in \cite[formula B] {fraenkel}:
\[ D^{\alpha}_w(f(g))(x) = \sum_{1\le |\beta|\le|\alpha| }D^{\beta}_w(f(g(x)) Q_{\alpha,\beta}(g,x)\]
In this formula $Q_{\alpha,\beta}(g,x)$ are homogeneous polynomials of degree $|\beta|\le l$ in the derivatives of order less than $l$ of the components of $g$. Moreover the coefficients of these polynomials depend only on $\alpha,l,n$. Hence there exists a constant $C$ depending only on $l,n,M$ such that $|Q_{\alpha,\beta}(g,x)|\ \le C$ uniformly on $x$. This concludes the proof.
\end{proof}

\begin{definition}[Sobolev Space]
Let $\Omega$ be an open set in $\rr^n$, $l \in \nn$ and $1\le p\le \infty$. We set
	\begin{align*} W^{l,p}(\Omega)=\{& f \in L^p(\Omega) \ | \ D^\alpha_w f \text{ exists and belongs to } L^p(\Omega)  \\
	 &\text{ for every } \alpha \in \nn^n_0 \text{ with } |\alpha|=l\}.
	\end{align*}
\end{definition}
The Sobolev space $W^{l,p}(\Omega)$ admits a natural norm given by
\[ \|f\|_{W^{l,p}(\Omega)}=\|f\|_{L^p(\Omega)}+\sum_{|\alpha|=l}\|D^\alpha_wf\|_{L^p(\Omega)}.\]
Equipped with the norm $\| .\|_{W^{l,p}(\Omega)}$, the space $W^{l,p}(\Omega)$ is a Banach space. \\\\

In the next definition we introduce the notion of the open set in $\rr^n$ with Lipschitz or $C^m$ boundary. Since the regularity of the boundary can be defined in different ways, we will use here the notion of \textit{set with resolved boundary} given in \cite[Section 4.3]{burenkov}.
\begin{definition}\label{resolved boundary}
Let $0<d\le D <\infty, M>0, \varkappa>0$ We say that an open set $\Omega$ in $\mathbb{R}^n$ has a resolved boundary with parameters $d,D, \varkappa$ if there exists a family of open cuboids $V_i , i=1,...,s$ (where $s \in \mathbb{N}$ if $\Omega$ is bounded and $s=\infty$ otherwise) such that
\begin{enumerate}
\item $(V_i)_d \cap \Omega \neq \emptyset$
\item $\Omega \subset \bigcup\limits_{j=1}^s (V_i)_d$
\item The multiplicity of the cover $\{ V_i\}_{i=1}^s$ is less than $ \varkappa$.
\item There exist isometries $\lambda_i$ of $\mathbb{R}^n$ such that
\[   \lambda_j(V_j)= \prod_{i=1}^n ]a_{ij},b_{ij}[\]
and, if $\partial V_j \cap \Omega \neq \emptyset$,
\[ \lambda_j(V_j\cap \Omega)= \{ (\overline x, x_n) \in \mathbb{R}^n  |  \overline x \in W_j , a_{nj}+d<x_n<\phi_j(\overline x)\} \]
where $W_j=\prod\limits_{i=1}^{n-1} ]a_{ij},b_{ij}[$ and $\phi_j : W_j \rightarrow \mathbb{R}$.

Moreover
\begin{itemize}
\item if $\phi_j \in C^m(\overline W_i)$ with  $\| D^{\alpha}\phi_j \|\le M <\infty $, for every $1\le|\alpha|\le m$, we say that $\Omega$  has a resolved $C^m$ boundary with parameters $d,D, \varkappa,M$.
\item if $\phi_j \in \Lip(\overline W_i)$ with  $\Lip(\phi)= M$,  we say that $\Omega$  has a resolved Lipschitz boundary with parameters $d,D, \varkappa,M$.
\end{itemize}
Finally we will say that a domain $\Omega$ has a resolved $C^m$ (or Lipschitz) boundary if there exist parameters $d,D,\varkappa,M$ for which $\Omega$ has a $C^m$ (or Lipschitz) boundary.
\end{enumerate}

\end{definition}
We give now the definition of Morrey spaces.
\begin{definition}
Let $1\le p< \infty$ , $\phi$ a function from $\rr^+$ to $\rr^+$ and $\Omega$ be a domain in $\mathbb{R}^n$. For a function $f \in L^p_{loc}(\Omega)$ we define the Morrey space as
\[ M_p^\phi(\Omega)=\{f \in  L^p_{loc}(\Omega) \ |\  \|f\|_{M_p^\phi(\Omega)}<\infty\}\]
where
\[ \|f\|_{M_p^\phi(\Omega)}:=\sup_{B_r(x), x \in \Omega,r>0} \left(  \frac{1}{\phi(r)}\int_{B_r(x)\cap \Omega} |f(y)|^p dy \right )^{\frac{1}{p}}.\]
\end{definition}
The space $M_p^\phi(\Omega)$ equipped with the norm $\|.\|_{M_p^\phi(\Omega)}$ is Banach space.
\begin{remark}
	Let $\Omega$ be an open set in $\rr^n$, and let $\phi(r)=r^\gamma$ with $\gamma>0.$ If $\gamma=n$, then $M_p^\phi(\Omega)=L^\infty(\Omega)$ and if $\gamma=0$ then $M_p^\phi(\Omega)=L^p(\Omega)$. If instead $\gamma>n$, then $M_p^\phi(\Omega)$ contains only the 0 function.
\end{remark}

We remark that definition we just gave is slightly more general than the classical definition of Morrey space, indeed usually $\phi(r)$ is taken to be just $r^\gamma$ for some $\gamma>0.$ We decided to do so because taking $r^\gamma$ instead of a general function $\phi(r)$ doesn't simplify much the proofs of the results that will appear in this work.\\
It will be useful to define also the following spaces, closely related to the Morrey spaces. 
\begin{definition}
Let $1\le p< \infty$ , $\phi$ a function from $\rr^+$ to $\rr^+$ and $\Omega$ be a domain in $\mathbb{R}^n$. For every $\delta>0$ and every function $f \in L^p_{loc}(\Omega)$ we define the norm $\|f\|_{M^{\delta,\phi}_p}$ as
\[\| f\|_{M^{\delta,\phi}_p(\Omega)} := \sup_{B_r(x), x \in \Omega, 0<r<\delta} \left(  \frac{1}{\phi(r)}\int_{B_r(x)\cap \Omega} |f(y)|^p dy \right )^{\frac{1}{p}}.\]
\end{definition}
\begin{remark}
	Let $\Omega$ be an open set in $\rr^n$. We observe that $M_p^{\delta,\phi}(\Omega)\subset M_p^\phi(\Omega) $ and $\|.\|_{M_p^{\phi,\delta}(\Omega)}\le\|.\|_{M_p^{\phi}(\Omega)}$ for any $\delta>0$ . Moreover if $\Omega$ ha finite diameter then $M_p^{\delta,\phi}(\Omega)= M_p^\phi(\Omega) $ and $\|.\|_{M_p^{\phi,\delta}(\Omega)}=\|.\|_{M_p^{\phi}(\Omega)}$for any $\delta\ge \text{diam}(\Omega).$ If instead $\Omega$ is unbounded we can still notice that, for any $f \in M_p^\phi(\Omega)$, the norm $\|f\|_{M_p^{\delta,\phi}(\Omega)}$ converges to the norm $\|f\|_{M_p^{\phi}(\Omega)}$ as $\delta$ goes to $+\infty.$ 
\end{remark}

\section{Hestenes Operator}
The first extension operator we will consider is the so called Hestenes operator. The main advantage of this operator is its simple construction, but the downside is that it requires a high regularity of the boundary. It was first published by Hestenes in \cite{fraenkel}. In \cite{fraenkel} this operator is used to extend functions of class $C^m(\Omega)$ to the whole space, where $\Omega$ is taken to be a closed set in $\rr^n$. This of course require to define the notion of a $C^m$ function in a closed set, which is done in \cite{fraenkel} following the previous work of Whitney [..]. However here we will focus on functions belonging to Sobolev spaces, hence we are interested to extend functions defined in open set $\Omega$ of $\rr^n.$ It turns out that the operator defined by Hestenes can be used also to extend functions from a Sobolev space $W^{l,p}(\Omega)$ to $W^{l,p}(\rr^n)$ given that $\Omega$ has a sufficiently regular boundary, i.e. of class $C^m$ with $m\ge l$. In the next section we will construct the Hestenes operator while showing its good behavior with respect to Sobolev spaces.

\subsection{Construction}
We construct the Hestenes operator for domains  $\Omega \subset \mathbb{R}^n$ with $C^m$ boundary mainly following paragraphs 6.2,6.3 of \cite{burenkov}. First we will consider a simple case where $\Omega$ is a $C^m$ half strip. 
\begin{lemma}\label{hestenes1}
Let $l,n,m \in \mathbb{N}, m\ge l, 1\le p \le \infty$ and $W = \prod\limits_{i=1}^{n-1}]a_i,b_i[$ be an open cuboid of $\mathbb{R}^{n-1}$. Moreover define
\[ S=W \times \mathbb{R} \]
\[\Omega =\{ (\overline x,x_n) | \overline x\in W, x_n<\phi(\overline x)  \} \]
where $\phi \in C^m(\overline W)$,$m\ge l$, and $\| D^{\alpha}\phi \|\le M <\infty $ for every $1\le|\alpha|\le l$. Then there exists a bounded extension operator $T$ from $W^{l,p}(\Omega)$ to $W^{l,p}(S).$
\end{lemma}





To prove Lemma $\ref{hestenes1}$ we prove first the case $\phi \equiv 0$ in the following result, that is a generalization of Lemma 9.2 in \cite{brezis}.

\begin{lemma}\label{hestenes2}
Let $l,n \in \mathbb{N}, 1\le p \le \infty$ and $W = \prod\limits_{i=1}^{n-1}]a_i,b_i[$ be an open cuboid of $\mathbb{R}^{n-1}$. There exists a bounded extension operator
\[ T: W^{l,p}(S^-) \rightarrow W^{l,p}(S)\]
where 
\[ S=W \times \mathbb{R} \]
\[ S^-=W \times \mathbb{R}^-.\]
\end{lemma}
\begin{proof}
Let $f \in W^{l,p}(S^-)$. We define
\[ Tf(\overline x,x_n) = \begin{cases}
		f(x), & \text {if }x_n<0, \\
		\sum_{k=1}^l \alpha_k f(\overline x,-\beta_k x_n),  &\text {if } x_n>0, 
		\end{cases}
\]
where $\alpha_k,\beta_k$ are real numbers that satisfy $\beta_k>0$ and
\begin{equation}\label{alfabeta}
 \sum_{k=1}^l \alpha_k (-\beta_k)^s=1  
\end{equation}
for every $s=0,...,l-1$.  Notice that given $\beta_1,...,\beta_l>0$  pairwise distinct, we can always find $\alpha_1,...,\alpha_l$ that satisfy the condition by solving a Vandermonde square system of linear equations. First we prove that $Tf \in W^{l,p}(S)$. We take any $\phi \in C^\infty_c(S)$ and consider the integral 
\[ \int_S Tf(x)D^\alpha\phi(x)dx= \int_{S^+} Tf(x)D^\alpha\phi(x)dx+ \int_{S^-} Tf(x)D^\alpha\phi(x)dx\]
where $S^+=\{(\overline x,x_n) \ | \  \overline x \in W, x_n>0 \}$ and $\alpha \in \mathbb{N}^n_0, 1\le|\alpha|\le l$. Let's write $\alpha=(\overline \alpha,\alpha_n)$, with $\overline \alpha \in \mathbb{N}^{n-1}_0$ and $\alpha_n \in  \mathbb{N}_0$.  By changing variables in the integrals we get 
\begin{align*}
 \int_S Tf(x)D^\alpha\phi(x)dx &= \int_{S^+}\sum_{k=1}^l \alpha_k f(\overline x,-\beta_k x_n)  D^\alpha\phi(x)dx+ \int_{S^-} f(x)D^\alpha\phi(x) dx\\
				   &= \int_{S^-} f(\overline y,y_n) D^\alpha \psi(\overline y, y_n) dy \tag{*}
\end{align*}
where $\psi(\overline x,x_n)= \sum_{k=1}^l -\alpha_k (-\beta_k)^{\alpha_n-1}\phi \left( \overline x,-x_n/\beta_k\right)+\phi(\overline x,x_n) $.Note that $\psi$ belongs to $\in C^\infty(S^-)$ but does not have compact support in $S^-$. To bypass this problem we use an auxiliary function $\nu \in C^\infty(\mathbb{R})$ that satisfies 
\[ \begin{cases}
	\nu(x)=0, & \text{if } x>-1/2, \\
	\nu(x)=1, & \text{if }x<-1,
\end{cases} 
\]
and we define the functions $\nu_k(t)=\nu(kt)$ for $k \in \mathbb{N}$. It's clear that $\psi(x) \nu_k(x_n) \in C^\infty_c(S^-)$, hence we can integrate by parts
\begin{equation}\label{parts}
\int_{S^-}  f(x)D^{\alpha}(\psi(x) \nu_k(x_n)) dx = (-1)^{|\alpha|}\int_{S^-} D^\alpha_w f(x)\psi(x) \nu_k(x_n) dx  
\end{equation}
By the Leibniz rule 
\begin{align*}
D^{\alpha}(\psi(x) \nu_k(x_n)) &=\frac{\partial^{\alpha_n}}{\partial x_n^{\alpha_n}} D^{\overline \alpha}(\psi(x) \nu_k(x_n)) \\
				        &= \nu(kx_n)D^{\alpha}\psi(x)+\sum_{i=1}^{\alpha_n}  {{\alpha_n}  \choose {i} }  k^i \nu^{(i)}(kx_n)\frac{\partial^{\alpha_n-i}}{\partial x_n^{\alpha_n-i}} D^{\overline \alpha}\psi(x).
\end{align*}
By the Dominated Convergence Theorem
\[ \int_{S^-} f(x) \nu(kx_n)D^{\alpha}\psi(x)dx  \rightarrow \int_{S^-} f(x) D^{\alpha}\psi(x)dx \text{    as } k\rightarrow \infty,\]
because $f \in L^1(S^- \cap \supp \psi)$ since $\supp \psi$ is bounded.
Next, we claim that for every $i=1,..,\alpha_n$
\begin{equation}\label{lim}
\int_{S^-}  f(x) k^i \nu^{(i)}(kx_n)\frac{\partial^{\alpha_n-i}}{\partial x_n^{\alpha_n-i}} D^{\overline \alpha}\psi(x)dx  \rightarrow 0 
\end{equation}
as $k \rightarrow \infty$. To prove this first we notice that since  $\alpha_k,\beta_k$ satisfies $\eqref{alfabeta}$ we have that 
\[   \frac{\partial^j}{\partial x_n^j} D^{\overline \alpha}\psi(\overline x, 0)=0 \text{  ;  }j=0,...,\alpha_n-1,\]
hence by Taylor formula 
\[  \left| \frac{\partial^{\alpha_n - i}}{\partial x_n^{\alpha_n - i}} D^{\overline \alpha}\psi(\overline x, x_n) \right|  \le \frac{C|x_n|^i}{i!},\]
for all $i=1,...,\alpha_n$, where $C= \sup_{x \in   S^-} |D^{\alpha} \psi(x)|$. Therefore we get the following estimate
\begin{align*}
   	\int_{S^-}  \left|  f(x) k^i \nu^{(i)}(kx_n)\frac{\partial^{\alpha_n-i}}{\partial x_n^{\alpha_n-i}} D^{\overline \alpha}\psi(x)     \right |  dx  &\le 
	 \frac{\widetilde CC}{i!} \int_{ \{ x \in S^-\cap \supp f \ , \ -1/k< x_n<0 \} }   |f(x)| k^i |x_n|^i  dx  \\
	& \le \frac{\widetilde CC}{i!}   \int_{ \{ x \in S^-\cap \supp f \ ,\  -1< x_n<0 \} }   |f(x)| dx 
\end{align*}
where $\widetilde C = \sup_\mathbb{R} |\nu^{(i)}|$. The second inequality comes from the fact that $\nu^{(i)}(x)=0$ for $x<-1$ and $i\ge 1$. Hence we get $\eqref{lim}$ by Dominated Convergence Theorem. Passing to the limit in $\eqref{parts}$ we obtain
\[ \int_{S^-}  f(x)D^{\alpha}\psi(x) dx = (-1)^{|\alpha|}\int_{S^-} D^\alpha_wf(x)\psi(x)  dx.  \]
which, combined with (*), implies
\[ \int_S Tf(x)D^{\alpha}\phi(x)dx=  \int_{S^-}  f(x)D^{\alpha}\psi(x) dx = (-1)^{|\alpha|}\int_{S^-} D^\alpha_wf(x)\psi(x)  dx. \]
Finally going back to the original coordinates and using the definition of $\psi$ we get
\begin{align*}
& \int_S Tf(x)D^{\alpha}\phi(x)dx = (-1)^{|\alpha|} \int_{S^-}D^\alpha_w f(x) \left[\sum_{k=1}^l -\alpha_k (-\beta_k)^{\alpha_n-1}\phi \left( \overline x,-\frac{x_n}{\beta_k} \right)+\phi(\overline x,x_n)\right] dx=\\
					&= (-1)^{|\alpha|} \int_{S^+} \sum_{k=1}^l \alpha_k (-\beta_k)^{\alpha_n} D^\alpha_w f(\overline y,-\beta_k y_n)\phi(y)dy +(-1)^{|\alpha|} \int_{S^-} D^\alpha_w f(y)\phi(y) dy
\end{align*}
that implies that $D^\alpha_wTf$ exists and 
\[ D^\alpha_wTf(x)= \begin{cases}
			 D^\alpha_w f(x),	& \text{if } x \in S^-, \\
			\sum_{k=1}^l \alpha_k (-\beta_k)^{\alpha_n} D^\alpha_w f(\overline x,-\beta_k x_n)\phi(x), & \text{if }x \in S^+.
		\end{cases}
 \]
It remains to prove the boundedness of  T. It's immediate to verify that
\[ \| Tf \|_{L^p(S^+)}\le \sum_{i=1}^l |\alpha_k|\beta_k^{-1/p} \| f\|_{L^p(S^-)} \] 
and that we have similar bounds for the norm of the weak derivatives of $Tf$. Hence there exists a constant $C$ depending only on $\beta_k,\alpha_k,l$ such that $\| Tf\|_{W^{l,p}(S^+)}\le C \| f\|_{W^{l,p}(S^-)} $. Observing that $\| Tf\|^p_{W^{l,p}(S)}=\| Tf\|^p_{W^{l,p}(S^+)}+\|f\|^p_{W^{l,p}(S^-)}$ the proof is concluded.
\end{proof}




\begin{proof}[Proof of Lemma $\ref{hestenes1}$ ]
Let $f \in W^{l,p}(\Omega).$  Consider the function $g$ from $S^-$ onto $\Omega$ defined by
\[  g(\overline x, x_n)= (\overline x , x_n+ \phi(\overline x))\]
for all $(\overline x, x_n) \in S^-$
and its inverse  $g^{-1}$
\[  g^{-1}(\overline x, x_n)= (\overline x , x_n- \phi(\overline x))\]
where $S^-=W \times \rr^-$. For all $f \in W^{l,p}(\Omega)$ we set
\begin{align*}
 Gf=f \circ g
\end{align*}
Since $g$ is a diffeomorphism between $S^-$ and $\Omega$ of class $C^m$, Lemma \ref{composition} guarantees that $Gf$ admits weak derivatives up to order $l$. We claim that $G$ defines a bounded operator from $W^{l,p}(\Omega)$ to $W^{l,p}(S^-)$, with bounded inverse. To prove this, first we compute the Jacobian matrix of $g^{-1}$

\[ Jg^{-1}(x)= \begin{bmatrix}
		1             &             0  &            \dots &            &0 \\
		0             & 1              &   \dots           &            & 0\\
		\vdots    &                 &  \ddots           &  & \vdots\\
		\vdots    &                 &           &\ddots    &\\
		-\frac{\partial \phi(\overline x)}{\partial x_1} &-\frac{\partial \phi(\overline x)}{\partial x_2}& \dots & \dots & 1 
	\end{bmatrix}
\]
from which $|\det(Jg^{-1}(x))| \equiv 1 $. Moreover, again by Lemma $\ref{composition}$, we have
\[ |D^{\alpha}_w(f(g))| \le C(l,M) \sum_{1\le |\beta|\le|\alpha| }|D^{\beta}_wf(g)|\]
where $C(l,M)$ depends only on $l$ and $M$, with $M= \sup_{1\le|\alpha|\le l} \| D^\alpha \phi \|$.
Next by the change of variable formula and Minkowski's inequality we get
\begin{align*}
 \left( \int_{S^-} |D^{\alpha}_w(f(g))(x)|^p dx \right)^{\frac{1}{p}} & \le \sum_{1\le |\beta|\le|\alpha| } C(l,M) \left( \int_{S^-} |D^{\beta}_wf(g(x))|^p  dx\right)^{\frac{1}{p}}  \\
										      &=  \sum_{1\le |\beta|\le|\alpha| } C(l,M)\left(  \int_\Omega |D^{\beta}_wf(y)|^p |\det Jg^{-1}\big|_{g(y)}|  dy\right)^{\frac{1}{p}}  \\
  											&=  \sum_{1\le |\beta|\le|\alpha| } C(l,M) \|D^{\beta}_wf \|_{L^p(\Omega)}					
\end{align*}
Thus, using the estimates for the intermediate derivatives, that
\[ \| Gf\|_{W^{l,p}(S^-)} =\| f(g)\|_{W^{l,p}(S^-)} \le C \| f\|_{W^{l,p}(\Omega)}\]
for a constant $C$ independent of $f$. In a similar way we can also prove that 
\[ \| G^{-1}f\|_{W^{l,p}(\Omega)}=\| f(g^{-1})\|_{W^{l,p}(\Omega)} \le D \| f\|_{W^{l,p}(S)}.\]
Now we can just define the operator $T$ as
\[  T=G^{-1} \circ \overline T \circ G\]
where $\overline T$ is the extension operator from $W^{l,p}(S^-)$ to $W^{l,p}(S)$ defined in Lemma $\ref{hestenes2}$. Therefore $T$ is bounded as composition of bounded operators. An explicit for for $T$ is

\[ Tf(x) = \begin{cases}
		f(x), & \text{if } x \in \Omega,\\
		\sum_{i=1}^l \alpha_k f(\overline x, \phi(\overline x)- \beta_k(x_n-\phi(\overline x))), & \text{if } x \in S \setminus \overline \Omega.
\end{cases} 
\]
\end{proof}


The following remark will be useful on the proof of Theorem \ref{teorh}.


\begin{remark}\label{hestenesr}{}
In the notation of Lemma $\ref{hestenes1}$, let $a,b \in \rr$ such that $a<\phi(\overline x) <b$ for every $\overline x \in W$. We define $S^{a,b}=W\times (a,b)$, $\Omega_a=\Omega \cap (W \times (a,\infty))$ and $\widehat W^{l,p}(\Omega_a)=\{f \in W^{l,p}(\Omega_a) | \supp f \subset S\}$ . Then exists a bounded extension operator
\[ T: \widehat W^{l,p}(\Omega_a)\rightarrow W^{l,p}(S^{a,b}).\]
To see this we can just extend $f\in \widehat W^{l,p}(\Omega_a)$ naturally by 0 to  $f_0\in W^{l,p}(\Omega)$ and then define
\[ Tf=(\widetilde T f_0)\big|_{S^{a,b}}\]
where $\widetilde T$ is the operator of the previous Lemmma.
\end{remark}
We can now consider the general case where $\Omega$ is a domain in $\mathbb{R}{}^n$ with a $C^m$ resolved boundary. This is done by considering a covering made of cuboids for $\Omega$, given by Definition \ref{resolved boundary}. Then Lemma \ref{hestenes1} is used to construct an extension operator for each cuboid and finally all these operators are attached together using a suitable partition of the unity. This is basically the scheme for the proof of the following result.

\begin{theorem}\label{teorh}
Let $m,l \in \mathbb{N}, l\le m$ and $1\le p\le\infty$. If $\Omega$ is a domain in $\mathbb{R}^n$ has a $C^m$ resolved boundary then there exists a bounded extension operator
\[ T: W^{l,p}(\Omega) \rightarrow W^{l,p}(\mathbb{R}^n).\]
\end{theorem}
\begin{proof}[Proof Sketch]
Let $f \in W^{l,p}(\Omega)$. Let $\{V_i\}_{i=1}^s$ be the covering of cuboids for $\Omega$ as in Definition \ref{resolved boundary}. It's possible to construct functions $\{ \psi_i \}_{i=1}^s \subset C^\infty_c(\mathbb{R}^n)$ such that the functions $\{ \psi^2_i \}_{i=1}^s$ form a partition of the unity corresponding to the covering $\{V_i\}_{i=1}^s$ and satisfying $\|D^\alpha \psi_i \|_{L^\infty}\le M_1$ with $M_1$ depending only on $n,l,d$.  If $\partial \Omega \cap V_i \neq \emptyset$ by Remark $\ref{hestenesr}$ there exists a bounded operator
\[ T_i : \widehat W^{l,p}(\lambda_i(\Omega \cap V_i))\rightarrow W^{l,p}(\lambda_i(V_i))\]
where $\widehat W^{l,p}(\lambda_i(V_i\cap \Omega)=\{f \in W^{l,p}(V_i\cap \Omega) | \supp f \subset \lambda_i(V_i)  \}.$ If $V_i \subset \Omega$ the operator $T_i$ is defined to be just the identity. We set
\[ Tf = \sum_{i=1}^s \psi_iT_i(\psi_if(\lambda_i^{-1}))(\lambda_i).\]
assuming ($\psi_iT_i(\psi_if(\lambda_i^{-1}))(\lambda_i))=0$ outside $V_i$. The functions $\psi_if \in W^{l,p}(V_i\cap \Omega)$ are such that $\supp \psi_if \subset \overline \Omega \cap V_i$, hence $\psi_i f(\lambda_i) \in \widehat W^{l,p}(\lambda_i(V_i\cap \Omega))$ and so $T$ is well defined. To see that $T$ is an extension operator, take $x \in \Omega$: if $x \in \supp \psi_i$ then $\psi_i(x)T_i(\psi_if(\lambda_i^{-1}))(\lambda_i(x))=\psi_i(x)^2 f(x)$ ; if $x \notin \supp \psi_i$ then $0=\psi_i(x)T_i(\psi_if(\lambda_i^{-1}))(\lambda_i(x))=\psi_i(x)^2 f(x)$. So $Tf(x)=\sum\limits_{i=1}^s \psi_i^2(x) f(x)=f(x)$. 

We omit the proof of the boundedness of $T$, the details of which can be found in the proofs of Lemma 13-14 in \cite{burenkov}.
\end{proof}

\subsection{Hestenes operator on Morrey spaces}

\
\begin{lemma}\label{covering}
Let $k\ge 1$ and $\Omega$ be set in $\rr^n$ with diameter $D>0$. Then there exists an integer $C_{n,k}$ depending only on $k$ and $n$ such that $\Omega$ can be covered by a collection of open balls $B_1,...,B_h$ centered in $\Omega$ with radius $D/k$ and $h \le C_{k,n}.$
\end{lemma}
\begin{proof}
We start by claiming that if $S$ is a set of points in $\rr^n$ satisfying
\begin{enumerate}[i)]
\item  	$S \subset \Omega$,
\item $\|z_1-z_2 \|\ge D/k$ for every $z_1,z_2 \in \Omega$ with $z_1 \neq z_2$,
\end{enumerate}
then $\#S\le C_{n,k}$ where $C_{k,n}$ is an integer depending only on $k$ and $n$. To see this, first note that $\Omega$ is contained in some closed cube $Q$ of side $2D$. Then we choose $m \in \nn$ such that $2^{m-1}>\sqrt n k$. Next we cover $Q$ with $(2^m)^n$ smaller closed cubes of side $2D/2^m$. The diagonal of a smaller cube measures $2D/2^m \cdot \sqrt n < D/k$. Thus each of these cubes can contain at most one point of $S$ ,so $\#S\le (2^m)^n.$ Therefore it's enough to choose $C_{n,k}=2^{mn}.$ Set $r:=D/k$, we'll prove that we can cover $\Omega$ with a collection of balls $B_1,...,B_h$ centered in $\Omega$ of radius $r$ and such that $k\le C_{n,k}.$ Choose $x_1 \in \Omega$ and take $B_1=B_r(x_1)$, the ball centered in $x_1$ of radius $r$. If $\Omega \subset B_1$ we are done, if not there exists $x_2 \in \Omega\setminus B_1$ and we take $B_2=B_r(x_2).$ Again, if $\Omega \subset (B_1\cup B_2)$ we stop, otherwise we can pick $x_3 \in  \Omega\setminus (B_1 \cup B_2)$ and take $B_3=B_r(x_3).$ We iterate this procedure : given $B_1,...,B_i$ balls, if $ \Omega \subset (B_1\cup...\cup B_i)$ we stop, otherwise we can choose $x_{i+1} \in  \Omega \setminus (B_1\cup...\cup B_i)$ and take $B_{i+1}=B_r(x_{i+1}).$ We claim that this procedure stops with $i\le C_{n,k}.$ Suppose it doesn't, then we can find $B_1,...,B_{C_{n,k}+1}$ balls centered respectively at $x_1,...,x_{C_{n,k}+1}$. Setting $S= \{x_1,...,x_{C_{n,k}+1} \}$, it's immediate to see that $S$ satisfies i) and ii), but $\#S=C_{n,k}+1$, that is a contradiction.
\end{proof}

\begin{lemma}\label{lcircle}
Let $W \subset \mathbb{R}^{n-1}$ be open connected and define
\[\Omega =\{ (\overline x,x_n) \ |\ \overline x\in W, x_n\le \psi(\overline x)  \} \]
\[\Omega^+ =\{ (\overline x,x_n) \ |\ \overline x\in W, x_n> \psi(\overline x)  \} \]
where $\psi \in \Lip(\overline W)$. Let $\beta>0$ and consider the function $A_\beta$ from $W \times \rr$ to $\Omega$ defined by
\[ A_\beta(\overline x, x_n) =\begin{cases}
							(\overline x , \psi(\overline x)-\beta(x_n-\psi(\overline x))), & \text{if } (\overline x, x_n) \in \Omega^+ ,\\
							(\overline x, x_n), & \text{if } (\overline x, x_n) \in \Omega .\\
\end{cases}
\]
 Then for every $x_0 \in W \times \mathbb{R}$ and $r>0$
\[ A(B_r(x_0)\cap \Omega^+)\subset B_{cr}(A(x_0))\cap \Omega\]
where $c\ge 1$ is a constant depending only on $\Lip \psi$ and $\beta$.
\end{lemma}

\begin{proof}
Notice that it is sufficient to prove that for every $x,y \in W \times \mathbb{R}$ we have
\begin{equation}\label{circle}
 \|A(x)-A(y)\| \le c\|x-y \|.
\end{equation}
Set $M= \Lip \psi$. We distinguish three cases: 1. $x,y \in \Omega$ : in this case $A(x)=x$ and $A(y)=y$, so $\|x-y \| = \|A(x)-A(y)\|$ and there is nothing to prove.

2. $x,y \in \Omega^+$ : we have 
\begin{align*}
|A(x)_n-A(y)_n|&=|\psi(\overline x)-\beta(x_n-\psi(\overline x))-\psi(\overline y)+\beta(y_n-\psi(\overline y))| \\
				& \le (1+\beta)|\psi(\overline x)-\psi(\overline y)| + \beta|x_n-y_n|\\
				& \le M(1+\beta)\|\overline x -\overline y\| + \beta|x_n-y_n|
\end{align*}
Hence
\begin{align*}
\|A(x)-A(y)\|^2 &=\|\overline {A(x)} -\overline {A(y)}\|^2+|A(x)_n-A(y)_n|^2\\
& \le \|\overline x -\overline y\|^2+[M(1+\beta)\|\overline x -\overline y\| + \beta|x_n-y_n|]^2 \\
& \le (1+2M^2(1+\beta)^2)\|\overline x -\overline y\|^2+2\beta^2|x_n-y_n|^2 \\
& \le c_1^2(M,\beta)\|x-y\|^2
\end{align*}
for some constant $c_1(M,\beta)$.

3. $x \in \Omega^+, y \in \Omega$ : first notice that, since $\psi(\overline x) < x_n$, then $x_n-y_n>\psi(\overline x)-y_n$. Moreover $\psi(\overline y) > y_n$, hence $M\|\overline x -\overline y\| \ge \psi(\overline y)-\psi(\overline x)> y_n-\psi(\overline x).$ This implies
\[ |\psi(\overline x)-y_n |< |x_n-y_n|+M\|\overline x -\overline y\|.\]
Now
\begin{align*}
|A(x)_n-A(y)_n|&=|\psi(\overline x)-\beta(x_n-\psi(\overline x))-y_n| \\
				&=|(1+\beta)(\psi(\overline x)-y_n)+\beta(y_n-x_n)|\\
				&\le M(1+\beta)\|\overline x -\overline y\|+(1+2\beta)|x_n-y_n|
\end{align*}
and
\begin{align*}
\|A(x)-A(y)\|^2 &=\|\overline {A(x)} -\overline {A(y)}\|^2+|A(x)_n-A(y)_n|^2\\
& \le \|\overline x -\overline y\|^2+[M(1+\beta)\|\overline x -\overline y\|+(1+2\beta)|x_n-y_n|]^2 \\
& \le (1+2M^2(1+\beta)^2)\|\overline x -\overline y\|^2+2(1+2\beta)^2|x_n-y_n|^2 \\
& \le c_2^2(M,\beta)\|x-y\|^2.
\end{align*}
for some constant $c_2(M,\beta)$. Then $\eqref{circle}$ by taking $c=max(\sqrt c_1,\sqrt c_2,1)$.

\end{proof}


\begin{lemma}\label{hmorrey1}
Let $l,n,m \in \mathbb{N},m\ge l, 1\le p \le \infty$, $W = \prod\limits_{i=1}^{n-1}]a_i,b_i[$ be an open cuboid of $\mathbb{R}^{n-1}$ and $\phi$ a function from $\rr^+$ to $\rr^+$. Moreover define
\[ S=W \times \mathbb{R} \]
\[\Omega =\{ (\overline x,x_n) | \overline x\in W, x_n<\psi(\overline x)  \} \]
where $\psi \in C^m(\overline W)$ and $\| D^{\alpha}\psi \|\le M <\infty $ for every $1\le|\alpha|\le l$. Then for every $f \in W^{l,p}(\Omega)$, $\delta>0$ and  $1\le |\alpha|\le l$
\begin{align}
\| Tf\|_{M_p^{\phi,\delta}(S)} &\le    C\| f\|_{M_p^{\phi,\delta}(\Omega)} ,\\
\| D^\alpha_w Tf\|_{M_p^{\phi,\delta}(S)} &\le   C\sum_{1\le |\beta|\le|\alpha| }\| D^\beta_wf\|_{M_p^{\phi,\delta}(\Omega)},
\end{align}
where $T$ is the Hestenes operator defined in Lemma $\ref{hestenes1}$ and $C$ is a constant independent of $f$.
\end{lemma}

\begin{proof}
Define $\Omega^+ =\{ (\overline x,x_n) \ |\ \overline x\in W, x_n> \psi(\overline x)  \}$ . We recall the definition of $T$
\[ Tf(x) = \begin{cases}
		f(x) & x \in \Omega \\
		\sum_{i=1}^l \alpha_k f(\overline x, \psi(\overline x)- \beta_k(x_n-\psi(\overline x))) & x \in \Omega^+
\end{cases} 
\]
and observe that we can rewrite it as
\[ Tf(x) = \begin{cases}
		f(x), & \text{if }x \in \Omega, \\
		\sum_{i=1}^l \alpha_k f(G_k(x)), & \text{if } x \in \Omega^+,
\end{cases} 
\]
where $G_k(\overline x, x_n)=(\overline x, \psi(\overline x)- \beta_k(x_n-\psi(\overline x)))$. Note that $G_k: \Omega^+\rightarrow \Omega$ defines a diffeomorphism from $\Omega^+$ to $\Omega$ of class $C^m$ and satisfying $|\det JG_k^{-1}|\equiv 1/\beta_k.$ First we prove ii). Let's fix $x_0 \in S$ and a radius $\delta>r>0$. We want to estimate the quantity 
\[ I=\left ( \frac{1}{\psi(r)} \int_{B_r(x_0)\cap S} |D^\alpha_w Tf(x)|^pdx \right)^{\frac{1}{p}}\]
for $1\le|\alpha| \le l$. To do this we estimate the integral as follows

\[ I \le \underbrace{\left ( \frac{1}{\phi(r)} \int_{B_r(x_0)\cap \Omega^+} |D^\alpha_w Tf(x)|^pdx \right)^{\frac{1}{p}}}_\textrm{$I_1$}+\underbrace{\left ( \frac{1}{\phi(r)} \int_{B_r(x_0)\cap \Omega} |D^\alpha_w Tf(x)|^pdx \right)^{\frac{1}{p}}}_\textrm{$I_2$}.\]
Since $Tf(x)=f(x)$ when $ x\in\Omega$, we have immediately  
\[ I_2 \le \|D^\alpha_w f \|_{M_p^{\phi,\delta}(\Omega)} . \]
It remains to estimate $I_1$. We start by observing that from Lemma \ref{composition} there exists a constant $C_k$ depending only on $G_k$ and $l$ such that
\[ |D^{\alpha}_w (f \circ  G_k)| \le C_k \sum_{1\le |\beta|\le|\alpha| }| D^{\beta}_wf(G_k)|.\]
By the previous inequality and Lemma \ref{lcircle} we are able to produce the following bound
\begin{align*}
 \frac{\|D^{\alpha}_w(f \circ G_k) \|_{L^p(B_r(x_0)\cap \Omega^+)}}{\phi(r)^{\frac{1}{p}}} &\le C_k \sum_{1\le |\beta|\le|\alpha| } \left(\phi(r)^{-1} \int_{G_k(B_r(x_0)\cap \Omega^+)} |D^\beta_wf(y)|^p|\det JG_k^{-1}\big|_{G_k(y)}|dy\right)^{\frac{1}{p}}\\
 		&\le C_k \beta_k^{-\frac{1}{p}} \sum_{1\le |\beta|\le|\alpha| } \left(\phi(r)^{-1} \int_{B_{c_kr}(A_{\beta_k}(x_0))\cap \Omega} |D^\beta_wf(y)|^pdy\right)^{\frac{1}{p}} 		\\		
\end{align*}
where $A_{\alpha_k}$ is defined as in Lemma \ref{lcircle} and $c_k$ depends only on $\beta_k$ and $M$. By Lemma \ref{covering} the set $B_{c_kr}(A_{\beta_k}(x_0))\cap \Omega$ can be covered with a collection of open balls $B_1,...,B_h$ centered in $\Omega$ with radius $r$ and $h\le m_k$, where $m_k$ depends only on $c_k$. Hence we get
\[ \frac{\|D^{\alpha}_w(f \circ G_k) \|_{L^p(B_r(x_0)\cap \Omega^+)}}{\phi(r)^{\frac{1}{p}}}\le C_k \beta_k^{-\frac{1}{p}} m_k \sum_{1\le |\beta|\le|\alpha| } \|D^\beta f\|_{M_p^{\delta,\phi}(\Omega)}\]
 Next we estimate $I_1$:
\begin{align*}
I_1 = \phi(r)^{-\frac{1}{p}}|D^\alpha_w Tf \|_{L^p(B_r(x_0)\cap \Omega^+)}&\le \phi(r)^{-\frac{1}{p}} \sum_{k=1}^l \alpha_k \|D^\alpha_w f(G_k) \|_{L^p(B_r(x_0)\cap \Omega^+)} \\
&\le \sum_{k=1}^l \alpha_kC_k\beta_k^{-\frac{1}{p}}m_k \left( \sum_{1\le |\beta|\le|\alpha| }\| D^\beta_wf\|_{M_p^{\phi,\delta}(\Omega)} \right).																	
\end{align*}
Finally putting the estimates of $I_1,I_2$ together 
\begin{align*}
\| D^\alpha_w Tf\|_{M_p^\phi(S)}&=\sup_{x_0 \in S, r>0} \left ( \frac{1}{\phi(r)} \int_{B_r(x_0)\cap S} |D^\alpha_w Tf(x)|^pdx \right)^{\frac{1}{p}} \\
									&\le \|D^\alpha_w f \|_{M_p^\phi(\Omega)}+\sum_{k=1}^l \alpha_kC_k\beta_k^{-\frac{1}{p}}m_k \left( \sum_{1\le |\beta|\le|\alpha| }\| D^\alpha_wf\|_{M_p^{\phi,\delta}(\Omega)} \right) \\
									& \le \widetilde C\sum_{1\le |\beta|\le|\alpha| }\| D^\alpha_wf\|_{M_p^{\phi,\delta}(\Omega)}
\end{align*}
where $\widetilde C$ depends only on $\{b_k\}_k,\{\alpha_k\}_k,l,M,p$. This proves ii). The proof of i) is exactly analogous to the proof of ii).
\end{proof}
		




\begin{theorem}
Let $m,l \in \mathbb{N}, l\le m$, $1\le p\le\infty$, $\phi$ a function from $\rr^+$ to $\rr^+$ and $\Omega$ a domain in $\mathbb{R}^n$ with $C^m$ resolved boundary. Let also $ T$ be the Hestenes operator defined in Theorem \ref{teorh}. Then if $\Omega$ is bounded, for every $f \in W^{l,p}(\Omega)$, $\delta>0$ and $1\le |\alpha| \le l$ we have
\begin{align}
\| Tf\|_{M_p^\phi(\rr^n)} &\le    C\| f\|_{M_p^\phi(\Omega)}, \label{b0}\\
\| D^\alpha_w Tf\|_{M_p^{\phi,\delta}(\rr^n)} &\le   C\sum_{1\le |\beta|\le|\alpha| }\| D^\beta_wf\|_{M_p^{\phi,\delta}(\Omega)}, \label{bl}
\end{align}
where  $C$ doesn't depend on $f$. If instead $\Omega$ is unbounded, for every $f \in W^{l,p}(\Omega)$ and $\delta>0$ we have
\begin{align}
\| Tf\|_{M_p^{\phi,\delta}(\rr^n)} &\le    C_\delta\| f\|_{M_p^\phi(\Omega)}, \label{u0} \\
\| D^\alpha_w Tf\|_{M_p^{\phi,\delta}(\rr^n)} &\le   C_\delta\sum_{1\le |\beta|\le|\alpha| }\| D^\beta_wf\|_{M_p^\phi(\Omega)}, \label{ul}
\end{align}
where $C_\delta$ depends on $\delta$ but not on $f.$
\end{theorem}
\begin{proof}
Let $f \in W^{l,p}(\Omega)$ and  $\{V_i\}_{i=1}^s$ be the covering of cuboids for $\Omega$ as in the definition of set with resolved boundary. We recall the definition of $T:$
\[ Tf = \sum_{i=1}^s \psi_iT_i(\psi_if(\lambda_i^{-1}))(\lambda_i)\]
where $\{ \psi^2_i \}_{i=1}^s$ form a partition of the unity corresponding to the covering $\{V_i\}_{i=1}^s$ and satisfying $\|D^\alpha \psi_i \|_{L^\infty}\le M_1$, with $|\alpha|\le l$ and $M_1$ depending only on $n,l,d$. To make the notation simpler we will rewrite $T$ as
\[ Tf = \sum_{i=1}^s \psi_i\widetilde T_i(\psi_if)\]
where the operator $\widetilde T_i$ is defined as $\widetilde T_i f = T_i(f(\lambda_i^{-1}))(\lambda_i)$. 
Before starting the proof we remark some facts that will be justified at the end:
\begin{enumerate}[a)]

\item Let $C_i$ the constant such that
\[ \| T_ig\|_{M_p^{\phi,\delta}(\lambda_i(V_i))}\le C_i \|g\|_{M_p^{\phi,\delta}(\lambda_i(\Omega \cap V_i))},\]
\[\| D^\alpha_w T_ig\|_{M_p^{\phi,\delta}(\lambda_i(V_i)))} \le   C_i\sum_{1\le |\beta|\le|\alpha| }\| D^\alpha_wg\|_{M_p^{\phi,\delta}(\lambda_i(\Omega \cap V_i)))},\]  

for $1\le |\alpha|\le l$, $g \in \widehat W^{l,p}(\lambda_i(\Omega \cap V_i))$ and $\delta>0$. Then $\sup_{i=1,...,s} C_i \le M_2$, where $M_2$ depends only on $\Omega,l,n$.
\item We have
\[ \| \widetilde T_ig\|_{M_p^{\phi,\delta}(V_i)}\le M_2 \|g\|_{M_p^{\phi,\delta}(\Omega \cap V_i)},\]
\[\| D^\alpha_w \widetilde T_ig\|_{M_p^\phi(V_i)} \le   M_3M_2\sum_{1\le |\beta|\le|\alpha| }\| D^\alpha_wg\|_{M_p^\phi(\Omega \cap V_i)},\]  
for $1\le |\alpha|\le l$,$g \in \widehat W^{l,p}(\Omega \cap V_i)$, $\delta>0$ and where $M_3$ doesn't depend on $i$. 
\end{enumerate}
Let now $x_0 \in \rr^n$, $0<r<\delta$ and $B_r(x_0)$ the ball centered in $x_0$ of radius $r.$ Let's consider the set $J=\{ i=1,...,s \ | \ V_i \cap B_r(x_0) \neq \emptyset\}.$
We notice that there exists an integer $\widetilde s$ depending only on the covering $(V_i)_{i=1}^s$ and on $\delta$ such that $\# J\le \widetilde s.$ We also recall that if $\Omega$ is bounded then $\widetilde s\le s<\infty$. We have



\begin{align*}
\left( \frac{1}{\phi(r)} \int_{B_r(x_0)} |Tf(x)|^pdx  \right)^{\frac{1}{p}} &=  \left( \frac{1}{\phi(r)} \int_{B_r(x_0)} |\sum_{i=1}^s \psi_i(x) \widetilde T_i(\psi_if))(x) |^pdx \right)^{\frac{1}{p}}\\
&\le \sum_{i \in J} \left( \frac{1}{\phi(r)} \int_{{B_r(x_0)} \cap V_i} | \widetilde T_i(\psi_if)(x)|^p |dx \right)^{\frac{1}{p}}\\
&\overset{b)}\le \widetilde s M_2\| \psi_i f\|_{M_p^{\phi,\delta}(V_i \cap \Omega)} \le M_2\widetilde s \| f\|_{M_p^{\phi,\delta}(\Omega)}.
\end{align*}
This proves \eqref{b0} and \eqref{u0}. Let now $\alpha \in \nn_0^n$ with $1\le|\alpha|\le l.$ We have 
\begin{align*}
\left( \frac{1}{\phi(r)} \int_{B_r(x_0)} |D^\alpha_w Tf(x)|^pdx  \right)^{\frac{1}{p}} &=  \left( \frac{1}{\phi(r)} \int_{B_r(x_0)} |D^\alpha_w \sum_{i=1}^s \psi_i(x) \widetilde T_i(\psi_if))(x) |^pdx \right)^{\frac{1}{p}}\\
&\le C_\alpha \sum_{i \in J} \left( \frac{1}{\phi(r)} \int_{{B_r(x_0)} \cap V_i} \sum_{\beta \le \alpha}|D^{\alpha -\beta}\psi_i(x) D^\beta_w \widetilde T_i(\psi_if)(x)|^p dx \right)^{\frac{1}{p}}\\
&\le C_\alpha M_1 \widetilde s \sum_{i \in J} \left( \frac{1}{\phi(r)} \int_{{B_r(x_0)} \cap V_i} \sum_{\beta \le \alpha}|D^\beta_w \widetilde T_i(\psi_if)(x)|^p dx \right)^{\frac{1}{p}}\\
&\overset{b)}\le C_\alpha M_1\widetilde s \sum_{\beta \le \alpha } M_2M_3 \sum_{|\gamma|\le |\beta|} \| D^\gamma_w f\|_{M_p^{\phi,\delta}(V_i)} \\
&  \le \widetilde C_\alpha M_1M_2M_3\widetilde s \sum_{|\beta| \le |\alpha|}\| D^\beta_w f\|_{M_p^{\phi,\delta}(V_i)} 
\end{align*}
This proves \eqref{bl} and \eqref{ul}. Let's now prove a) and b). a) $\Omega$ has a resolved $C^m$ boundary with parameters $\varkappa,d,D,M$. Hence, if $\phi_i$ are the $C^m$ functions of Definition 1, we have $\| D^{\alpha}\phi_i \|_{L^\infty}\le M$ for every $i$ and for every $1\le|\alpha|\le l$. Therefore by the proof of Lemma \ref{hmorrey1} we deduce that $C_i$ depends only on $l,n,M$ and on the choice of the constants $\alpha_k,\beta_k$, which can be chosen to be the same for every $T_i$.
b) We notice that since $\lambda_i$ are isometries, they are smooth and their derivatives are uniformly bounded with a bound depending only on $n$. Then the result follows from a) and from a straightforward computation using a change of variable and Lemma \ref{composition}.
\end{proof}
	
\section{Stein operator}

\subsection{Construction}
In this section we will define the Stein extension operator for Lipschitz domains in $\rr^n$. The details of the construction and the proofs of all the results in this subsection can be found in \cite[Section 2-3, Ch. VI]{stein}. We start by introducing the notion of regularized distance with the following theorem. Here by $d(x,F)$ we denote the distance of a point $x \in \rr^n$ from the set $F\subset \rr^n.$ 

\begin{theorem}\label{regdist}
Let $F$ be a closed set in $\rr^n$. Then there exists a real-valued function $\Delta(.)=\Delta(.,F)$ defined in $F^c$ such that
\begin{enumerate}[a)]
\item $c_1 d(x,F)\le \Delta(x)\le c_2d (x,F),$ for every $x \in F^c,$  
\item $\Delta$ is $C^\infty$ in $F^c$ and
\[ \left | D^\alpha \Delta(x) \right | \le B_\alpha d(x,F)^{1-|\alpha|},\]
\end{enumerate}
for every $x \in F^c$, where $B_\alpha$, $c_1$,$c_2$ are constants independent of $x$ and $F$.
\end{theorem}
Next we give the definition of a special Lipschitz domain.
\begin{definition}
A domain $\Omega$ of $\rr^n$ is said to be a special Lipschitz domain if there exists a Lipschitz function $\psi$ defined from $\rr^{n-1}$ to $\rr$ such that
\[ \Omega=\{(\overline x, y) \in \rr^n \ | \ \psi(\overline x)<y \}.\]
Moreover the Lipschitz constant $\Lip \psi$ is said to be the Lipschitz bound of $\Omega.$ 
\end{definition}
It is convenient to define first the Stein extension operator in the case of a special Lipschitz domain. To do so we need the following two lemmas.
\begin{lemma}\label{lemma1}
Let $\Omega$ be a special Lipschitz domain of $\rr^n$ and set $F=\overline \Omega$. Let $\Delta$ be the regularized distance from $F$ as given in Theorem \ref{regdist}. Then there exists a positive constant $a$, which depends only on the Lipschitz bound of $\Omega$, such that if $(\overline x, y) \in F^c$, then $a\Delta(\overline x,y)\ge \psi(\overline x)- y.$
\end{lemma}

\begin{lemma}\label{lemma2}
There exists a continuous real-valued function $\tau$ defined in $[1,\infty)$ satisfying
\begin{enumerate}[i)]
	\item $\tau(\lambda)=O(\lambda^N)$, as $\lambda \rightarrow \infty$ for every $N,$
	\item $\int_1^\infty \tau(\lambda)d\lambda=1$, $\int_1^\infty \lambda^k\tau(\lambda)d\lambda=0$, for every $k=1,2,...$
\end{enumerate}
\end{lemma}


\begin{theorem}\label{defT}Let $\Omega$ be a special Lipschitz domain of $\rr^n$ with Lipschitz bound $M$. Moreover let $\tau$ be the function in Lemma \ref{lemma2} and $a$ the constant of Lemma \ref{lemma1}. For every function $f$ that is $C^\infty$ in $\overline \Omega$ and bounded in $\overline \Omega$ together with all its partial derivatives, define
\begin{equation} Tf(\overline x, y)= \begin{cases}
						f(\overline x, y), & \text{ if } y\ge\psi(\overline x) \\
						\int_1^\infty f(\overline x, y+ \lambda \delta^*(\overline x,y))\tau(\lambda)d\lambda, & \text{ if } y<\psi(\overline x),		
\end{cases}
\label{defT2}\end{equation}
where $\delta^*(\overline x,y)=2a\Delta(\overline x, y).$ Then $Tf \in C^\infty(\rr^n)$ and 
\[\| Tf\|_{W^{l,p}(\rr^n)}\le C_{n,l}(M) \| f\|_{W^{l,p}(\Omega)} ,\]
where $C_{l,n}(M)$ is a constant depending only on $n,l$ and $M.$
\end{theorem}

We are now ready to define the Stein extension operator in the case of special Lipschitz domains. The construction is the following. Let $\Omega$ be a special Lipschitz domain in $\rr^n$ with Lipschitz bound $M$. We denote by $\Gamma$ the cone with vertex at the origin given by $\Gamma=\{(\overline x, y) \in \rr^n \ | \ M |\overline x|<|y|, y<0 \}$. Suppose now that $\eta \in C^\infty_c(\rr^n)$ is a non-negative function with integral 1 and which support is contained in $\Gamma.$ For every $f \in W^{l,p}(\Omega)$ and every $\varepsilon>0$ we define
\[f_\varepsilon(x)=\frac{1}{\varepsilon^n}\int_{\rr^n} f(x-y) \eta(y/\varepsilon)dy =\int_{\rr^n} f(x-\varepsilon y) \eta(y)dy.\]
Notice that, since the support of $\eta$ is strictly inside $\Gamma$, the above integral is well defined for every $x$ in some neighborhood of $\overline \Omega$ depending on $\varepsilon$. Hence $f_\varepsilon \in C^\infty(\overline\Omega)$ and it is bounded with all its partial derivatives, thus $Tf_\varepsilon$ is well defined. The Stein operator is then taken to be the limit of $Tf_\varepsilon$ as $\varepsilon \to 0.$ This limit procedure is formalized in the following result.

\begin{theorem}\label{Sdef}
Let $l \in \nn,1\le p \le \infty$ and $\Omega$ be a special Lipschitz domain of $\rr^n$ with Lipschitz bound $M$. For every $f \in W^{l,p}(\Omega)$ define $Tf_\varepsilon$ as in \eqref{defT2}. Then $Tf_\varepsilon $ converges in $W^{l,p}(\rr^n)$ if $p<\infty$ and in $W^{l-1,p}(\rr^n)$ if $p=\infty$, as $\varepsilon \to 0.$ Moreover setting
\[ Sf=\lim_{\varepsilon \to 0} Tf_\varepsilon\]
we have that $Sf$ extend $f$ to $\rr^n$ and 
\[ \| Sf\|_{W^{l,p}(\rr^n)} \le C_{l,n}(M) \| f\|_{W^{l,p}(\Omega) } ,\] 
where $C_{l,n}(M)$ is a constant depending only on $n,l$ and $M.$
\end{theorem}

\begin{remark}\label{rotlip}
Let $\Omega$ be a domain in $\rr^n$ and suppose that there exists a rotation $R$ of $\rr^n$ such that $R(\Omega)$ is a special Lipschitz domain with Lipschitz bound $M$. We observe that we can use the operator $S$ to extend the space $W^{l,p}(\Omega)$ to $W^{l,p}(\rr^n)$ continuously. Indeed, given $f \in W^{l,p}(\Omega)$, by Lemma \ref{composition} we have $f\circ R^{-1} \in W^{l,p}(R(\Omega))$. Hence we can use Theorem \ref{Sdef} to extend  $f\circ R^{-1}$ to $\rr^n$ with $S(f\circ R^{-1})\in W^{l,p}(\rr^n).$ Then $S(f\circ R^{-1})\circ R$ clearly extends $f$ and $S(f\circ R^{-1})\circ R \in W^{l,p}(\rr^n)$ by Lemma \ref{composition}. Now given $\alpha \in \nn^n_0$ with $|\alpha|\le l$ we argue as follows. Applying repeatedly \eqref{compbound} we have
\begin{align*}
&\left( \int_{\rr^n} |D^\alpha_w(S(f\circ R^{-1})\circ R)(x)|^pdx \right)^\frac{1}{p} \le C \sum_{|\beta|\le |\alpha|} \left( \int_{\rr^n} |D^\beta_w(S(f\circ R^{-1}))(R)|^p dx\right)^\frac{1}{p}=\\
&= C \sum_{|\beta|\le |\alpha|} \left( \int_{\rr^n} |D^\beta_w(S(f\circ R^{-1}))|^p |\det JR^{-1}(x)| dx\right)^\frac{1}{p}\\
&= C \sum_{|\beta|\le |\alpha|} \left( \int_{\rr^n} |D^\beta_w(S(f\circ R^{-1}))|^p  dx\right)^\frac{1}{p} \\
&\le CC_{l,n}(M)  \sum_{|\beta|\le |\alpha|} \sum_{|\gamma|\le |\beta|}\left( \int_{\rr^n} |D^\gamma_w(f\circ R^{-1})|^p  dx\right)^\frac{1}{p}\\
&\le C^2C_{l,n}(M)  \sum_{|\beta|\le |\alpha|} \sum_{|\gamma|\le |\beta|} \sum_{|\eta|\le |\gamma|}\left( \int_{\rr^n} |D^\eta_wf(R^{-1})|^p  dx\right)^\frac{1}{p}=\\
&= C^2C_{l,n}(M)  \sum_{|\beta|\le |\alpha|} \sum_{|\gamma|\le |\beta|} \sum_{|\eta|\le |\gamma|}\left( \int_{\rr^n} |D^\eta_w f|^p  dx\right)^\frac{1}{p},
\end{align*}
where $C$ depends only on the bound of the derivatives of $R$, hence only on $n$. This proves the continuity of the extension. In what follows we will denote the extension operator for a rotated special Lipschitz domain, that is $S(f\circ R^{-1})\circ R$, just by $Sf$.
\end{remark}

\begin{definition}\label{minsmooth}
Let $\Omega$ be an open set in $\rr^n$ and let $\partial \Omega$ be its boundary. We say that $\partial \Omega$ is minimally smooth if there exists an $\varepsilon >0$, $N \in \nn$, $M>0$ and a sequence $\{U_i\}_{i=1}^s$ (where $s$ can be $+\infty$) of open sets such that:
\begin{enumerate}[i)]
\item if $x \in \partial \Omega,$ then $B_\varepsilon(x) \subset U_i,$ for some $i$, where $B_\varepsilon(x)$ is the open ball centered in $x$ of radius $\varepsilon.$
\item No point of $\rr^n$ is contained in more than $N$ elements of the family $\{U_i\}_{i=1}^s.$
\item For every $i=1,...,s$ there exist a special Lipschitz domain $D_i$ and a rotation $R_i$ of $\rr^n$ such that
\[ U_i\cap \Omega = U_i \cap R_i(D_i).\]
\item The Lipschitz bound of $D_i$ does not exceed $M$ for every $i$. 
\end{enumerate}
\end{definition}

We now give the outline of the construction of the Stein extension operator for a set with minimally smooth boundary. The details of this construction and the proof of Theorem \ref{Eteor} can be found in \cite{stein}. 

First we introduce the following notation: given a set $U$ in $\rr^n$ and  $\varepsilon >0$ we set $U_\varepsilon=\{ x 	\in U \ | \ B_\varepsilon(x) \subset U\}$. Now let $\Omega$ be an open set in $\rr^n$ with minimally smooth boundary $\partial \Omega$. Consider also the constants $\varepsilon,N,M$ and the sequence of open sets  $\{U_i\}_{i=1}^s$ relative to $\Omega$ as given in Definition \ref{minsmooth}. We can construct a sequence of real-valued functions $\{\lambda_i\}_{i=1}^s$ defined in $\rr^n$, such that
\begin{itemize}
	\item $\supp \lambda_i \subset U_i$ for every $i=1,...,s$,
	\item $-1\le \lambda_i\le 1$,
	\item $\lambda_i(x)=1$ for every $x \in U_{\varepsilon/2},$
	\item every $\lambda_i$ is of class $C^\infty$, has bounded derivatives of all orders and the bounds of the derivatives of $\lambda_i$ can be taken to be independent of $i.$
\end{itemize}
We can also construct two real-valued functions $\Lambda_+,\Lambda_-$ defined in $\rr^n$, that satisfy the following conditions
\begin{itemize}
	\item $\supp \Lambda_+ \subset   \{ x \in \Omega \ | \ d(x,\partial \Omega)\le \varepsilon \}\cup \{ x \in \rr^n \ | \ d(x,\partial \Omega)\le \varepsilon /2\},$
	\item $\supp \Lambda_- \subset \Omega,$
	\item $|\Lambda_+|,|\Lambda_-|\le 1$
	\item $\Lambda_++\Lambda_- =1 $ in $\overline \Omega,$
	\item $\Lambda_+,\Lambda_-$ are of class $C^\infty(\rr^n)$ with bounded derivatives of all orders.
\end{itemize}
Consider now the extension operators $S_i$ for $W^{l,p}(R_i(D_i))$, defined as in Remark \ref{rotlip}. We define the extension operator $E$ for $\Omega$ as follows
\begin{equation} 
Ef(x):= \Lambda_+(x)\frac{\sum_{i=1}^s\lambda_i(x)S_i(\lambda_if)(x)}{\sum_{i=1}^s \lambda^2_i(x)}+\Lambda_-(x)f(x). \label{defEE}
\end{equation}

\begin{theorem}\label{Eteor}
Let $1\le p\le \infty, l,n \in \nn$. Let $\Omega$ be an open set in $\rr^n$ having minimally smooth boundary. Then $E$ is an extension operator which maps $W^{l,p}(\Omega)$ continuously into $W^{l,p}(\rr^n).$
\end{theorem}
\subsection{Stein operator in Sobolev-Morrey spaces}

\begin{definition}
Let $x$ be a point in $\rr^n$ and $r>0$. We define the open cube centered in $x$ of side $l$ as the set 
\[ Q_l(x)=(x_1-l/2,x_1+l/2) \times (x_2-l/2,x_2+l/2) \times \cdots \times (x_n-l/2,x_n+l/2)   \]
where $x=(x_1,...,x_n).$
\end{definition}


\begin{definition}
Let $1\le p< \infty$, $\phi$ a function from $\rr^+$ to $\rr^+$ and $\Omega$ be a domain in $\mathbb{R}^n$. For a function $f \in L^p_{loc}(\Omega)$ and $\delta>0$ we define the norm $\| .\|_{M_{p,Q}^{\phi,\delta}(\Omega)}$ as
\[ \|f\|_{M_{p,Q}^{\phi,\delta}(\Omega)}:=\sup_{\substack{Q_{2r}(x) \\ x \in \Omega \\\delta>r>0}} \left(  \frac{1}{\phi(r)}\int_{Q_{2r}(x)\cap \Omega} |f(y)|^p dy \right )^{\frac{1}{p}}\]
where $Q_{2r}(x)$ is the open cube centered in $x$ of side $2r$.
\end{definition}
\begin{lemma}\label{cubicmorrey}
Let $1\le p\le \infty$ , $\phi$ a function from $\rr^+$ to $\rr^+$ and $\Omega$ be a domain in $\mathbb{R}^n$. Then then norm $\| .\|_{M_{p,Q}^\phi(\Omega)}$ is equivalent to the classical Morrey norm $\|.\|_{M_{p}^\phi(\Omega)}$. In particular 
\[ \| .\|_{M_{p}^{\phi,\delta}(\Omega)} \le \| .\|_{M_{p,Q}^{\phi,\delta}(\Omega)}\le C_{n}\| .\|_{M_{p}^{\phi,\delta}(\Omega)}\]
where $C_n$ is a constant depending only on $n.$
\end{lemma}

\begin{proof}
We prove first the second inequality of the statement. Let $x \in \Omega$, $\delta>r>0$, $Q_{2r}(x)$ be the cube centered in $x$ of side $2r$ and $f \in L^p_{loc}(\Omega)$. Since the set $Q_{2r}(x) \cap \Omega$ has diameter less than $2r\sqrt n$ by Lemma \ref{covering} there exists a collection of balls $B_1,...,B_k$ centered in $Q_{2r}(x) \cap \Omega$ of radius $r$, with $k\le C_n$ where $C_n$ depends only on $n.$ Hence 
\[ 	\int_{Q_{2r}(x)\cap\Omega} |f(y)|^p dy \le \sum_{i=1}^k \int_{B_i\cap\Omega} |f(y)|^p dy\]
and
\[ \| f\|_{M_{p,Q}^{\phi,\delta}(\Omega)}=\sup_{Q_{2r}(x) ,x \in \Omega,r>0} \left( \frac{1}{\phi(r)}\int_{Q_{2r}(x)\cap\Omega} |f(y)|^p dy \right)^{\frac{1}{p}} \le C_n \| f\|_{M_p^{\phi,\delta}(\Omega)}.\]
To prove the first inequality we observe that for every $x \in \Omega$ and $r>0$, $(B_r(x)\cap\Omega)\subset (Q_{2r}(x)\cap\Omega)$, where $Q_{2r}(x)$ is the cube centered in $x$ with side $2r$ and $B_r(x)$ is the ball of radius $r$ centered in $x$. Therefore for every $f \in L^p_{loc}(\Omega)$
\[ \int_{B_r(x)\cap\Omega} |f(y)|^p dy \le \int_{Q_{2r}(x)\cap\Omega} |f(y)|^p dy\]
and this concludes the proof.

\end{proof}

\begin{lemma}\label{derivatives}
Let $\Omega$ be an open set in $\rr^n$ and let $f,h \in C^\infty(\rr^n)$. Define the function $g\in C^\infty(\rr^n)$ by $g(x)=f(\overline x, x_n+\lambda h(x))$ where $\overline x=x_1,...,x_{n-1}$ and $0\neq\lambda \in \rr.$ Then, for every $\alpha \in \nn_0^n$ and $x \in \rr^n$, $D^\alpha g(x)$ is a finite sum of terms of the following form

\[
c\lambda^s D^{\beta} f(\overline x, x_n+\lambda h(x))(D^{\gamma_1}h(x))^{n_1}\cdots (D^{\gamma_k}h(x))^{n_k}
\]
for some constant $c$, with $\beta,\gamma_i \in \nn_0^n $, $k,s,n_i \in \nn_0$ and $\beta,\gamma_i\neq0$, $k,s\ge 0$, $n_i>0$. It is meant that for $k=0$ no term $(D^{\gamma_i}h(x))^{n_i}$ is present. Moreover every term satisfies the following conditions
\begin{enumerate}[a)]
\item $n_1(|\gamma_1|-1)+n_2(|\gamma_2|-1)+...+n_k(|\gamma_k|-1)=|\alpha|-|\beta|$,
\item  $s=0$ if and only if $k=0$.
\end{enumerate}
\end{lemma}

\begin{proof}
We will prove the result by induction on $l=|\alpha|.$ Let's prove the case $l=1$. For every $i=1,...,n$ we have
\[ \frac{\partial g}{\partial x_i}(x)=\frac{\partial f}{\partial x_i}(\overline x, x_n+\lambda h(x))+\lambda\frac{\partial f}{\partial x_n}(\overline x, x_n+\lambda h(x))\frac{\partial h}{\partial x_i}(x) \]
that clearly satisfies the statement. We assume now that the result is true for $l$, and suppose $|\alpha|=l+1$. We write $D^\alpha g(x)=\frac{\partial D^\beta g}{\partial x_i}(x)$ for some $|\beta|=l$. Hence by induction hypothesis and linearity of the derivative we have that $D^\alpha g(x)$ is a finite sum of terms of the form
\[ \frac{\partial}{\partial x_i} [c\lambda^s D^{\gamma} f(\overline x, x_n+\lambda h(x))(D^{\gamma_1}h(x))^{n_1}\cdots (D^{\gamma_k}h(x))^{n_k}].\]
Suppose first that $k\ge 1,$ so by induction we know that
\begin{equation}\label{indder}
 n_1(|\gamma_1|-1)+n_2(|\gamma_2|-1)+...+n_k(|\gamma_k|-1)=|\beta|-|\gamma| 
 \end{equation}
 and that $s\ge 1.$
 Now using the chain rule we get
\begin{align*}
&\frac{\partial}{\partial x_i} [c\lambda^s D^{\gamma} f(\overline x, x_n+\lambda h(x))(D^{\gamma_1}h(x))^{n_1}\cdots (D^{\gamma_k}h(x))^{n_k}]=\\
&=c\lambda^s \frac{\partial D^{\gamma} f}{\partial x_i}  (\overline x, x_n+\lambda h(x))(D^{\gamma_1}h(x))^{n_1}\cdots (D^{\gamma_k}h(x))^{n_k}+\\
&+ c\lambda^{s+1} \frac{\partial D^{\gamma} f}{\partial x_n}  (\overline x, x_n+\lambda h(x))(D^{\gamma_1}h(x))^{n_1}\cdots (D^{\gamma_k}h(x))^{n_k}\frac{\partial h}{\partial x_i} (x)+\\
&+ \sum_{j=1}^k c\lambda^sn_j D^{\gamma} f(\overline x, x_n+\lambda h(x))(D^{\gamma_1}h(x))^{n_1}\cdots (D^{\gamma_k}h(x))^{n_k} \frac{\frac{\partial D^{\gamma_j}h}{\partial x_i} (x)}{D^{\gamma_j}h(x)} \addtag \label{derivate}.
\end{align*}
Let's see that every term in the right hand side of \eqref{derivate} satisfies a). By \eqref{indder} we have
\[ n_1(|\gamma_1|-1)+n_2(|\gamma_2|-1)+...+n_k(|\gamma_k|-1)=|\beta|-|\gamma|=|\alpha|-|\gamma + e_i|\]
where $e_i=(0,...,1,...,0),$ is the $n$-th element of the canonical base of $\rr^n$. Hence that first summand satisfies a). Again by \eqref{indder}
\[ n_1(|\gamma_1|-1)+n_2(|\gamma_2|-1)+...+n_k(|\gamma_k|-1)+(|e_i|-1)=|\alpha|-|\gamma + e_n|\]
and this proves a) for the second term. Now we consider the final sum, we will prove a) just for $j=1$, since the other terms can be discussed in the same way. We need to prove that
\[ n_1(|\gamma_1|-1)+...+(n_j-1)(|\gamma_j|-1)+...+n_k(|\gamma_k|-1)+(|\gamma_j+e_i|-1)=|\alpha|-|\gamma|. \]
Expanding the left-hand side we get
\[ n_1(|\gamma_1|-1)+n_2(|\gamma_2|-1)+...+n_k(|\gamma_k|-1)+1\]
and since $|\beta|=|\alpha|-1$ we conclude using \eqref{indder}. We observe that, since $k,s\ge1$, all the terms also satisfies b). \\
Suppose know that $k=0$, hence we need to consider
\[ \frac{\partial}{\partial x_i} [c D^{\gamma} f(\overline x, x_n+\lambda h(x)) ]\]
that becomes

\[ c \frac{\partial D^{\gamma} f}{\partial x_i} (\overline x, x_n+\lambda h(x)) + c\lambda \frac{\partial D^{\gamma} f}{\partial x_n} (\overline x, x_n+\lambda h(x)) \frac{\partial h}{\partial x_i}(x). \]
By induction and by a) we know that $|\gamma|=|\beta|,$ therefore it's immediate that both the above terms satisfies a) and b).

\begin{remark}\label{deltastar}
Let $\Omega$ be a special Lipschitz domain and let $\delta^*(\overline x, y)$ be the function defined in Theorem \ref{defT}. Then for every $(\overline x, y)$ with $\psi(\overline x)>y$ the following holds
\[c(\psi(\overline x)-y) \ge \delta^*(\overline x, y) \ge 2 (\psi(\overline x)-y), \] 
where $c$ is some constant depending only on $n.$ The second inequality follows directly from the definition of $\delta^*$ and Lemma \ref{lemma1}. Next we notice that $(\psi(\overline x)-y)\ge d(x,\overline \Omega)$, hence the first inequality follows from a) of Theorem \ref{regdist}.
\end{remark}

\end{proof}
\begin{lemma}\label{Tlemma}
Let $1\le p<\infty,n\ge2$, $\phi$ a function from $\rr^+$ to $\rr^+$ and $\Omega$ be a special Lipschitz domain of $\rr^n$ with Lipschitz bound $M.$ Moreover let $T$ be the operator defined in Theorem \ref{defT} and $f \in C^\infty(\overline \Omega)$ be a function bounded in $\overline \Omega$ together with all its partial derivatives. Then  for every $\alpha \in \nn_0^n$ and $\delta>0$
\begin{equation}
 \| D^\alpha Tf\|_{M_p^{\phi,\delta}(\rr^n)}\le C_{l,n}(M)\sum_{|\beta|\le |\alpha|}\|D^\beta f \|_{M_p^{\phi,\delta}(\Omega)} \label{Tbound2}
 \end{equation}
where $l=|\alpha|$ and $C_{l,n}(M)$ is a constant depending only on $l,n$ and $M.$
\end{lemma}
\begin{proof}
Let's start by proving the case $l=0$. By Lemma \ref{cubicmorrey} it's enough to prove that for an arbitrary open cube $Q$ of side $0<r<\delta$ in $\rr^n$ with sides parallel to the axis we have
\begin{equation}
\left(\frac{1}{\phi(r/2)}\int_Q |Tf(x)|^pdx \right)^{\frac{1}{p}} \le C_n(M) \| f\|_{M_{p,Q}^{\phi,\delta/2}(\Omega)} \label{cubestimate}
\end{equation}
for a constant $C_n(M)$ depending only on $n,M$. Let's define  $\Omega^- = \{ (\overline x , y) \in \rr^n \ | \ \overline x \in \rr^{n-1}, \ y<\psi(\overline x) \}$. There are three cases: 1. $Q \subset \Omega$ 2. $Q \subset \Omega^-$ 3. $Q\cap \{y=\psi(\overline x)\} \neq \emptyset.$ 

Case 1. Since $Tf=f$ in $\Omega$
\[ \left(\frac{1}{\phi(r/2)}\int_Q |Tf(x)|^pdx \right)^{\frac{1}{p}}=\left(\frac{1}{\phi(r/2)}\int_Q |f(x)|^pdx \right)^{\frac{1}{p}} \le  \| f\|_{M_{p,Q}^{\phi,\delta/2}(\Omega)}\]
and we are done.

Case 2. Let's write $Q$ as $Q=\{ (\overline x,y) \in \rr^n \ | \ \overline x \in F, y \in (a-r,a) \}$ where $F$ is an open cube of $\rr^{n-1}$ of side $r$ and $a<\psi(\overline x)$ for every $\overline x \in F$. Fix now $(\overline x, y) \in Q$. By Lemma \ref{lemma2} there exists a constant $A_3$ such that $|\tau(\lambda)|\le A_3/\lambda^3$  for every $\lambda \ge 1.$ From the definition of $Tf$ we have
\begin{equation} |Tf(\overline x,y)| \le\int_1^\infty |f(\overline x, y+\lambda \delta^*(\overline x,y))||\tau(\lambda)|d\lambda \le A_3 \int_1^\infty |f(\overline x, y+\lambda \delta^*(\overline x,y))|\frac{1}{\lambda^3}d\lambda \label{bullet1}
\end{equation}
Let's apply the change of variable $s=y+\lambda \delta^*(\overline x,y)$
\begin{equation}  |Tf(\overline x,y)|\le A_3\int_{y+\delta^*}^\infty |f(\overline x, s)|\frac{(\delta^*)^2}{(s-y)^3}ds\le A_3 c^2 \int_{2\psi(\overline x)-y}^\infty |f(\overline x, s)|\frac{(\psi(x)-y)^2}{(s-y)^3}ds \label{bullet2}
\end{equation} 
because $c(\psi(x)-y)\ge\delta^*\ge 2(\psi(x)-y)$ as seen in Remark \ref{deltastar}. Let's now decompose the last integral as follows
\[ |Tf(\overline x,y)|\le \sum_{k=0}^\infty A_3c^2\int_{2\psi(\overline x)-y+kr}^{2\psi(\overline x)-y+(k+1)r} |f(\overline x, s)|\frac{(\psi(\overline x)-y)^2}{(s-y)^3}ds.\]
Now by applying Minkowski's inequality for an infinite sum we get
\begin{align*} &\left(\int_{a-r}^{a}|Tf(\overline x,y)|^p dy\right)^{\frac{1}{p}} \\
&\le A_3 c^2\sum_{k=0}^\infty \left( \int_{a-r}^{a}\left ( \int_{2\psi(\overline x)-y+kr}^{2\psi(\overline x)-y+(k+1)r} \frac{|f(\overline x, s)|(\psi(x)-y)^2}{(s-y)^3}ds\right)^pdy \right)^{\frac{1}{p}} \addtag \label{sum2} 
\end{align*}
Next we plan to estimate each summand in \eqref{sum2}. To each summand in the right-hand side of \eqref{sum2} we apply the change of variable $y=\psi(\overline x)-z$ and we get
\[ \left( \int_{\psi(x)-a}^{\psi(x)-a+r}\left (\int_{\psi(x)+z+kr}^{\psi(x)+z+(k+1)r} |f(\overline x, s)|\frac{z^2}{(s-\psi(x)+z)^3}ds\right)^pdz \right)^{\frac{1}{p}}\]
and the change of variable $u=s-\psi(x)$
\[ \left( \int_{\psi(x)-a}^{\psi(x)-a+r}\left (\int_{z+kr}^{z+(k+1)r} |f(\overline x, u+\psi(x))|\frac{z^2}{(u+z)^3}du\right)^pdz \right)^{\frac{1}{p}}.\]
Then we apply the change of variable $t=u/z$
\[ \left( \int_{\psi(\overline x) - a}^{\psi(\overline x) -a+r}\left (\int_{1+kr/z}^{1+(k+1)r/z} |f(\overline x, tz+\psi(x))|\frac{1}{(t+1)^3}dt\right)^pdz \right)^{\frac{1}{p}}.\]
that can be rewritten as
 \[  \left( \int_{\psi(\overline x) -a}^{\psi(\overline x) -a+r}\left (\int_{1+kr/(\psi(\overline x) -a+r)}^{1+(k+1)r/(\psi(\overline x)-a)} |f(\overline x, tz+\psi(x))|\mathbbm{1}_{(1+kr/z,  1+(k+1)r/z ) }(t)\frac{1}{(t+1)^3}dt\right)^pdz \right)^{\frac{1}{p}}.\]
 By Minkowsi's integral inequality and setting $\alpha=r/(\psi(\overline x)-a)$
 \begin{align*}
  & \left( \int_{a\psi(\overline x) -a}^{\psi(\overline x) -a+r}\left (\int_{1+k\alpha/(\alpha+1)}^{1+(k+1)\alpha} |f(\overline x, tz+\psi(x))|\mathbbm{1}_{(1+kr/z,  1+(k+1)r/z ) }(t)\frac{1}{(t+1)^3}dt\right)^pdz \right)^{\frac{1}{p}} \\
  &\le \int_{1+k\alpha/(\alpha+1)}^{1+(k+1)\alpha} \left ( \int_{\psi(\overline x) -a}^{\psi(\overline x) -a+r}|f(\overline x, tz+\psi(x))|^p\mathbbm{1}_{(1+kr/z,  1+(k+1)r/z ) }(t)\frac{1}{(t+1)^{3p}} dz \right) ^{\frac{1}{p}}dt.  
 \end{align*}
 We notice that for every $t,z \in \rr$ with $\psi(\overline x) -a \le z \le \psi(\overline x)-a+r$
 \[ \mathbbm{1}_{(1+kr/z,  1+(k+1)r/z ) }(t) \le \mathbbm{1}_{(\psi(\overline x)-a+kr, \psi(\overline x) - a+(k+2)r)}(tz)   \]
 hence using the change of variable $w=tz$
 \begin{align*}
 &\int_{1+k\alpha/(\alpha+1)}^{1+(k+1)\alpha} \left ( \int_{\psi(\overline x) -a}^{\psi(\overline x) -a+r}|f(\overline x, tz+\psi(x))|^p\mathbbm{1}_{(1+kr/z,  1+(k+1)r/z ) }(t)\frac{1}{(t+1)^{3p}} dz \right) ^{\frac{1}{p}}dt \\
&\le \int_{1+k\alpha/(\alpha+1)}^{1+(k+1)\alpha} \left ( \int_{\psi(\overline x)-a+kr}^{\psi(\overline x)-a+(k+2)r}|f(\overline x, w+\psi(\overline x))|^p \frac{1}{t(t+1)^{3p}} dw \right) ^{\frac{1}{p}}dt\\
&=\int_{1+k\alpha/(\alpha+1)}^{1+(k+1)\alpha}\frac{1}{t^{\frac{1}{p}}(t+1)^{3}}dt \left ( \int_{\psi(\overline x)-a+kr}^{\psi(\overline x)-a+(k+2)r}|f(\overline x, w+\psi(\overline x))|^p  dw \right) ^{\frac{1}{p}}\\
&\le \int_{1+k\alpha/(\alpha+1)}^{1+(k+1)\alpha}\frac{1}{(t+1)^{3}}dt \left ( \int_{\psi(\overline x)-a+kr}^{\psi(\overline x)-a+(k+2)r}|f(\overline x, w+\psi(\overline x))|^p  dw \right)^{\frac{1}{p}}\\
&\le \int_{1+k\alpha/(\alpha+1)}^{1+(k+2)\alpha}\frac{1}{(t+1)^{3}}dt \left ( \int_{\psi(\overline x)-a+kr}^{\psi(\overline x)-a+(k+2)r}|f(\overline x, w+\psi(\overline x))|^p  dw \right)^{\frac{1}{p}}\\
&= \frac{1}{2}\left[\frac{1}{(2+k\alpha/(\alpha+1))^2} -\frac{1}{(2+(k+2)\alpha)^2}\right] \left ( \int_{\psi(\overline x)-a+kr}^{\psi(\overline x)-a+(k+2)r}|f(\overline x, w+\psi(\overline x))|^p  dw \right) ^{\frac{1}{p}}\\
&=\frac{s_k(\alpha)}{2} \left ( \int_{\psi(\overline x)-a+kr}^{\psi(\overline x)-a+(k+2)r}|f(\overline x, w+\psi(\overline x))|^p  dw \right) ^{\frac{1}{p}}.
\end{align*}
Where $s_k(\alpha)=\frac{1}{(2+k\alpha/(\alpha+1))^2} -\frac{1}{(2+(k+2)\alpha)^2}.$ Plugging this estimate inside \eqref{sum2} we get
\begin{align*}
 \left(\int_{a-r}^{a}|Tf(\overline x,y)|^p dy\right)^{\frac{1}{p}} &\le A_3\frac{c^2}{2} \sum_{k=0}^\infty s_k(\alpha) \left ( \int_{\psi(\overline x) -a+kr}^{\psi(\overline x) -a+(k+2)r}|f(\overline x, w+\psi(\overline x))|^p  dw \right) ^{\frac{1}{p}} \\
&= A_3\frac{c^2}{2} \sum_{k=0}^\infty s_k(\alpha) \left ( \int_{2\psi(\overline x) -a+kr}^{2\psi(\overline x) -a+(k+2)r}|f(\overline x, y)|^p  dy \right) ^{\frac{1}{p}}. \addtag \label{sum3}\\
\end{align*}
Taking the $L^p$ norm on $F$ on both sides and applying again Minkowski inequality we obtain
\begin{align*}
\left(\int_F\int_{a-r}^{a}|Tf(\overline x,y)|^p dy d\overline x\right)^{\frac{1}{p}} &\le A_3 \frac{c^2}{2} \sum_{k=0}^\infty s_k(\alpha) \left (\int_F \int_{2\psi(\overline x) -a+kr}^{2\psi(\overline x) -a+(k+2)r}|f(\overline x, y)|^p  dy d\overline x\right) ^{\frac{1}{p}}\\
&=A_3 \frac{c^2}{2} \sum_{k=0}^\infty s_k(\alpha) \|f\|_{L^p(S_k)}. \addtag \label{finalsum}\\
\end{align*}
where $S_k=\{ (\overline x, y) \in \rr^n \ | \ \overline x \in F ,\  2\psi(\overline x) -a+kr < y < 2\psi(\overline x) -a+(k+2)r \}$. The set $S_k$ has the following two properties 
\begin{enumerate}[a)]
 	\item $S_k$ has diameter less than $dr$, where $d$ is a constant depending only on $n$ and $M$.
 	\item $S_k \subset \Omega$.
 \end{enumerate}  To prove a), let $(\overline x_1,y_1),(\overline x_2,y_2)$ be two arbitrary points in $S_k$. We can suppose that $y_2\ge y_1.$ Then 
\begin{align*}
|y_1-y_2|&=y_2-y_1 \\
		&\le2\psi(\overline x_2)-a+(k+2)r -(2\psi(\overline x_1) -a+kr) \\
		&=2 (\psi(\overline x_2)-\psi(\overline x_1))+2r\le 2M|\overline x_1 - \overline x_2|+2r.
\end{align*}
Moreover 
\[ |\overline x_1 - \overline x_2|\le r \sqrt {n-1}\]
because $\overline x_1,\overline x_2 $ belongs to the $n-1$-dimensional cube $F.$ This proves a). To prove b) just notice that for every $(\overline x, y ) \in S_k$ we have $y>2\psi(\overline x)-a>\psi(\overline x)$.
Property a) together with Lemma \ref{covering} implies that there exists a collection of open cubes $Q_1,...,Q_m$ centered in $S_k$ of side $r$ that covers $S_k$, with $m \in \nn$ depending only on $M$ and $n$. Hence
\[ S_k \subset \bigcup_{i=1}^m (Q_i\cap \Omega) \]
and property b) assures that every cube $Q_i$ is centered in $\Omega.$ Therefore by \eqref{finalsum} 
\[ \| Tf\|_{L^P(Q)} \le \frac{A_3c^2}{2}\sum_{k=0}^\infty s_k(\alpha) (\|f\|_{L^p(Q_1\cap \Omega)}+...+\|f\|_{L^p(Q_m\cap \Omega)}),\]
 then dividing in both sides by $\phi(r/2)^{\frac{1}{p}}$ we obtain
\[\left(\frac{1}{\phi(r/2)}\int_Q|Tf(x)|^p dx\right)^{\frac{1}{p}} \le \frac{A_3c^2m}{2} \sum_{k=0}^\infty s_k(\alpha) \| f\|_{M^{\phi,\delta/2}_{p,Q}(\Omega)} \]
We want now to estimate the series $\sum_{k=0}^\infty s_k(\alpha)$. First we rewrite it in the following way
\begin{align*}
\sum_{k=0}^\infty s_k(\alpha)&=\sum_{k=0}^\infty \frac{1}{(2+k\alpha/(\alpha+1))^2} -\frac{1}{(2+(k+2)\alpha)^2}=\\
&=\sum_{k=0}^\infty \frac{(\alpha+1)^2}{(2+(k+2)\alpha)^2} -\frac{1}{(2+(k+2)\alpha)^2}=\\
&=\sum_{k=0}^\infty \frac{\alpha(\alpha+2)}{(2+(k+2)\alpha)^2}= \sum_{k=2}^\infty \frac{\alpha(\alpha+2)}{(2+k\alpha)^2}.
\end{align*}
To bound this series we distinguish two cases, when $\alpha\le 1$ and when $\alpha>1$. In the first case we can bound the series using a Riemann Sum
\begin{align*}
\sum_{k=2}^\infty \frac{\alpha(\alpha+2)}{(k\alpha+2)^2} &\le 3\sum_{k=2}^\infty \frac{\alpha}{(k\alpha+2)^2}=\\
&=3\sum_{k=2}^\infty \int_{\alpha (k-1)}^{\alpha k} \frac{1}{(\alpha k+2)^2}dt \le 3\int_0^\infty \frac{1}{(t+2)^2}dt=\frac{3}{2}. 
\end{align*}
In the second case
\[  \sum_{k=2}^\infty \frac{\alpha(\alpha+2)}{(k\alpha+2)^2}  \le \sum_{k=2}^\infty \frac{\alpha(\alpha+2)}{k^2\alpha^2}=\sum_{k=2}^\infty \frac{1+\frac{2}{\alpha}}{k^2} \le 3 \bigl(\frac{\pi^2}{6}-1\bigr)<2.\]
Hence we get 
\[\left(\frac{1}{\phi(r/2)}\int_Q|Tf(x)|^p dx\right)^{\frac{1}{p}} \le \frac{3mA_3c^2}{2} \| f\|_{M_{p,Q}^{\phi,\delta/2}(\Omega)} \]
that shows \eqref{cubestimate}. 

Case 3. We write $Q$ as $F \times (a-r,a)$ and and we define $Q^+=Q\cap\Omega$ and $Q^-=Q\cap\Omega^-.$ Then 
\[\|Tf\|_{L^p(Q)}\le\|f\|_{L^p(Q^+)}+\|Tf\|_{L^p(Q^-)}.\]
Moreover $Q^-$ can be furtherly decompose as $Q^-=Q^-_1 \cup Q^-_2$ where $Q^-_1=\{ (\overline x,y) \in Q^- \ | \ \psi(\overline x)>a \}$ and $Q^-_2=\{ (\overline x,y) \in Q^- | \psi(\overline x)\le a \}$. Hence
\begin{align*} \int_{Q^-} |Tf(x)|^pdx &= \int_{Q^-_1} |Tf(x)|^pdx +\int_{Q^-_2} |Tf(x)|^pdx\\
&=\int_{S_1} \int_{a-r}^{a} |Tf(\overline x,y)|^pdy d\overline x+\int_{S_2} \int_{a-r}^{\psi(\overline x)} |Tf(\overline x,y)|^pdy d\overline x
\end{align*}
for two suitable measurable sets $S_1$ and $S_2$ with $S_1 \cup S_2 = F.$ From \eqref{sum3} we know that if $\overline x \in S_1$ then
\begin{align*}
 \left(\int_{a-r}^{a}|Tf(\overline x,y)|^p dy\right)^{\frac{1}{p}} \le A_3\frac{c^2}{2} \sum_{k=0}^\infty s_k(\alpha) \left ( \int_{2\psi(\overline x) -a +kr}^{2\psi(\overline x) -a +(k+2)r}|f(\overline x, y)|^p  dy \right) ^{\frac{1}{p}}. \\
\end{align*}
Hence taking the $L^p$ norm over $S_1$ and reasoning as in Case 2 we obtain
\begin{equation}
\frac{1}{\phi(r/2)^{\frac{1}{p}}} \| Tf\|_{L^p(Q^-_1)} \le c_1 \| f\|_{M_p^{\phi,\delta/2}(\Omega)} \label{q1}
\end{equation}
for some constant $c_1$ depending only on $n$ and $M$. If instead $\overline x \in S_2$, since $\psi(\overline x)\le a$, we have
\begin{equation}
\int_{a-r}^{\psi(\overline x)} |Tf(\overline x,y)|^pdy \le \int_{\psi(\overline x)-r}^{\psi(\overline x)} |Tf(\overline x,y)|^pdy. \label{psia}
\end{equation}
Now from \eqref{sum3} with $a=\psi(\overline x)-\delta$ ($\delta>0$) we obtain
\begin{align*}
 \left(\int_{\psi(\overline x)-\delta-r}^{\psi(\overline x)-\delta}|Tf(\overline x,y)|^p dy\right)^{\frac{1}{p}} \le A_3\frac{c^2}{2} \sum_{k=0}^\infty s_k(\alpha) \left ( \int_{\psi(\overline x) +\delta +kr}^{\psi(\overline x) +\delta +(k+2)r}|f(\overline x, y)|^p  dy \right) ^{\frac{1}{p}}. \\
\end{align*}
Taking this time the $L^p$ norm in $S_2$ 
\begin{align*}
\left(\int_{S_2}\int_{\psi(\overline x)-\delta-r}^{\psi(\overline x)-\delta}|Tf(\overline x,y)|^p dy d\overline x\right)^{\frac{1}{p}} &\le A_3 \frac{c^2}{2} \sum_{k=0}^\infty s_k(\alpha) \left (\int_{S_2} \int_{\psi(\overline x) +\delta +kr}^{\psi(\overline x) +\delta +(k+2)r}|f(\overline x, y)|^p  dy d\overline x\right) ^{\frac{1}{p}}\\
&=A_3 \frac{c^2}{2} \sum_{k=0}^\infty s_k(\alpha) \|f\|_{L^p(S'_k)}. \\
\end{align*}
One can observe that the sets $S'_k$ have the properties a) and b) like the sets $S_k$ in Case 2, therefore
\[  \left(\frac{1}{\phi(r/2)}\int_{S_2} \int_{\psi(\overline x)-\delta-r}^{\psi(\overline x)-\delta}|Tf(\overline x,y)|^p dy d\overline x\right)^{\frac{1}{p}}\le c_2 \| f\|_{M_p^{\phi,\delta/2}(\Omega)} \]
for some constant $c_2$ depending only on $n$ and $M$. We now let $\delta$ go to 0 
\begin{equation}
  \left(\frac{1}{\phi(r/2)}\int_{S_2} \int_{\psi(\overline x)-r}^{\psi(\overline x)}|Tf(\overline x,y)|^p dy d\overline x\right)^{\frac{1}{p}}\le c_2 \| f\|_{M_p^{\phi,\delta/2}(\Omega)} . \label{q2}
 \end{equation}
Combining the above inequality with \eqref{psia} we obtain
\[  \left(\frac{1}{\phi(r/2)}\int_{S_2} \int_{a-r}^{\psi(\overline x)}|Tf(\overline x,y)|^p dy d\overline x\right)^{\frac{1}{p}}\le c_2 \| f\|_{M_p^{\phi,\delta/2}(\Omega)} .\]
Thus from \eqref{q1} and \eqref{q2}
\[ \frac{1}{\phi(r/2)^{\frac{1}{p}}} \| Tf\|_{L^p(Q^-)} \le \frac{1}{\phi(r/2)^{\frac{1}{p}}} \| Tf\|_{L^p(Q^-_1)} +\frac{1}{\phi(r/2)^{\frac{1}{p}}} \| Tf\|_{L^p(Q^-_2)} \le (c_1 +c_2)\| f\|_{M_p^{\phi,\delta/2}(\Omega)} \]
 Finally it's immediate to verify that $\| f\|_{L^p(Q^+)} \le \phi(r/2)^{\frac{1}{p}} \| f\|_{M_p^{\phi,\delta/2}(\Omega)}.$ This concludes the proof of Case 3.

We consider now the case $l>0.$ By Lemma \ref{cubicmorrey} it's again enough to prove that for an arbitrary open cube $Q$ of side $r$ contained in $\rr^n$ we have
\begin{equation}
\left(\frac{1}{\phi(r/2)}\int_Q |D^\alpha Tf(x)|^pdx \right)^{\frac{1}{p}} \le C_{l,n}(M) \sum_{|\beta| \le |\alpha|}\| D^\beta f\|_{M_{p,Q}^{\phi,\delta/2}(\Omega)}
\end{equation}
for a constant $C_{l,n}(M)$ depending only on $l,n,M$. We will consider the same three cases that appeared with $l=0$. Since $D^\alpha Tf=D^\alpha f$ in $\Omega$, the first case is trivial as before. We will see that the cases 2 and 3 also follow from the computations done with $l=0$. We start observing that by the boundedness of $f$ and all its derivatives we can differentiate under the integral sign to get
\[D^\alpha Tf(\overline x,y)= \int_1^\infty D^\alpha g_\lambda(\overline x,y) \tau(\lambda) d\lambda\]
for every $(\overline x, y) \in \Omega^-$, where $g_\lambda(\overline x,y)=f(\overline x, y+\lambda \delta^*(\overline x, y))$. By Lemma \ref{derivatives} $D^\alpha g_\lambda(\overline x,y)$ is a finite sum of terms of the type 
\[ \widetilde c\lambda ^s D^\beta f(\overline x, y+\lambda \delta^*(\overline x, y)(D^{\gamma_1}\delta^*(x))^{n_1}\cdots (D^{\gamma_k}\delta^*(x))^{n_k}.\]
For each of these terms we also set
\begin{align*}
 &T_{s,\beta,(\gamma_1,n_1),...,(\gamma_k,n_k)}(x) \\
 &= \int_1^\infty \lambda ^s D^\beta f(\overline x, y+\lambda \delta^*(\overline x, y)(D^{\gamma_1}\delta^*(x))^{n_1}\cdots (D^{\gamma_k}\delta^*(x))^{n_k} \tau(\lambda) d\lambda.
 \end{align*}
In this way $D^\alpha Tf(\overline x,y)$ is a finite sum of terms of type $\widetilde c T_{s,\beta,(\gamma_1,n_1),...,(\gamma_k,n_k)}(x)$. Now, since the constants $\widetilde c$ and the number of terms of the sum depend only on $l$ and $n$, we just need to estimate the quantities
\[\left( \frac{1}{\phi(r/2)}\int_Q  \left| T_{s,\beta,(\gamma_1,n_1),...,(\gamma_k,n_k)}(x)\right|^p dx\right )^{\frac{1}{p}}.\]
We start by assuming that $|\beta|=|\alpha|.$ By the property a) in Lemma \ref{derivatives} and by the estimates of the derivatives of $\delta^*(=2a\Delta)$ given in Theorem \ref{regdist} we have that

\begin{align*}
 |T_{s,\beta,(\gamma_1,n_1),...,(\gamma_k,n_k)}(x) | &\le c_3 \int_1^\infty\lambda^s |D^\beta f(\overline x, y+\lambda \delta^*(\overline x, y)| |\tau(\lambda)|d\lambda \\
&\le c_3A_{s+3} \int_1^\infty|D^\beta f(\overline x, y+\lambda \delta^*(\overline x, y)| \frac{1}{\lambda^3}d\lambda
\end{align*}
where  $A_{s+3}$ is such that $|\tau(\lambda)|\le A_{s+3}/\lambda^{s+3}$ and  $c_3$ depends only  on $n$ and $M.$ We are now in the same situation as in the second inequality of \eqref{bullet1}. Hence we can proceed the estimate in the same way as in case $l=0$ to get 
\[\left( \frac{1}{\phi(r/2)}\int_Q  \left| T_{s,\beta,(\gamma_1,n_1),...,(\gamma_k,n_k)}(x)\right|^p dx\right )^{\frac{1}{p}} \le c_4 \| D^\beta f \|_{M_p^{\phi,\delta/2}(\Omega)} \] 
for every $Q$ in case 2 and
\[\left( \frac{1}{\phi(r/2)}\int_{Q \cap \Omega^-}  \left| T_{s,\beta,(\gamma_1,n_1),...,(\gamma_k,n_k)}(x)\right|^p dx\right )^{\frac{1}{p}} \le c_5 \| D^\beta f \|_{M_p^{\phi,\delta/2}(\Omega)} \] 
 for every $Q$ in Case 3, where $c_4,c_5$ depend only on $n$ and $M.$
Suppose now that $|\alpha|>|\beta|.$ Arguing as above, by Theorem \ref{regdist} and Lemma \ref{derivatives} we get
\begin{align*}
&|T_{s,\beta,(\gamma_1,n_1),...,(\gamma_k,n_k)}(x) | \\
&\le c_6 \frac{1}{d(x,\overline \Omega)^{|\alpha|-|\beta|}} \left|\int_1^\infty \lambda^sD^\beta f(\overline x, y+\lambda \delta^*(\overline x, y)\tau(\lambda)d\lambda \right| \\
& \le  c_6 \frac{1}{(\psi(\overline x)-y)^{|\alpha|-|\beta|}} \left |\int_1^\infty \lambda^s D^\beta f(\overline x, y+\lambda \delta^*(\overline x, y) \tau(\lambda)d\lambda \right |\addtag \label{zzz} .   
\end{align*}
Where $c_6$ depends only on $n,l$ and $M$. We now write the Taylor expansion with integral remainder of the function $t \mapsto D^\beta f(\overline x, y+t)$ centered in $\delta^*(\overline x,y)$ up to order $m=|\alpha|-|\beta|$ and evaluated at $\lambda \delta^*(\overline x,y)$
\[
D^\beta f(\overline x, y+\lambda \delta^*) =\sum_{i=0}^{m-1} \frac{(\lambda \delta^*-\delta^*)^i}{i!}\frac{\partial^i D^\beta f}{\partial x_n^i}(\overline x,y+\delta^*) +\int_{\delta^*}^{\lambda \delta^*} \frac{(\lambda \delta^*-t)^{m-1}}{m!}\frac{\partial^{m} D^\beta f}{\partial x_n^{m} }(\overline x,y+t)dt. 
\]
We observe that the terms inside the first sum in the right hand side don't give any contribution in \eqref{zzz}, since
\begin{align*} &\int_1^\infty \frac{\lambda^s(\lambda \delta^*-\delta^*)^i}{i!}\frac{\partial^i D^\beta f}{\partial x_n^i}(\overline x,y+\delta^*)\tau(\lambda)d\lambda \\
&=\frac{\partial^i D^\beta f}{\partial x_n^i}(\overline x,y+\delta^*) \frac{(\delta^*)^i}{i!} \int_1^\infty \lambda^s(\lambda-1)^i\tau(\lambda)d\lambda=0
\end{align*}
by the properties of $\tau$, since $s>0$ by Lemma \ref{derivatives}. Hence combining this with \eqref{zzz} we obtain
\begin{align*}
&|T_{s,\beta,(\gamma_1,n_1),...,(\gamma_k,n_k)}(x) | \\
&\le \frac{c_6}{(\psi(\overline x)-y)^{m}} \left |\int_1^\infty \int_{\delta^*}^{\lambda \delta^*} \frac{(\lambda \delta^*-t)^{m-1}}{m!}\frac{\partial^{m} D\beta f}{\partial x_n^{m} }(\overline x,y+t)dt \lambda^s \tau(\lambda)d\lambda\right |.
\end{align*}
Observing that $(\lambda\delta^*-t)^{m-1}\le (\lambda\delta^*)^{m-1}$, recalling that $\psi(\overline x)-y\ge c \delta^*$ and using the change of variable $u=y+t$ we get
\begin{align*}
|T_{s,\beta,(\gamma_1,n_1),...,(\gamma_k,n_k)}(x) | \le \frac{c_6}{c^mm!\delta^*}\int_1^\infty \int_{y+\delta^*}^{y+\lambda \delta^*} \left|\frac{\partial^{m} D\beta f}{\partial x_n^{m} }(\overline x,u)\right |\lambda^{s+m-1} |\tau(\lambda)|du d\lambda.
\end{align*}
Performing a change of order of integration we deduce
\[ |T_{s,\beta,(\gamma_1,n_1),...,(\gamma_k,n_k)}(x) | \le\frac{c_6}{c^mm!\delta^*}\int_{y+\delta^*}^{\infty} \left |\frac{\partial^{m} D^\beta f}{\partial x_n^{m} }(\overline x,u)\right | \int_{(u-y)/\delta^*}^\infty |\lambda^{s+m-1} \tau(\lambda)| d\lambda du.\]
Finally recalling that that  $|\tau(\lambda)|\le A_{m+s+3}/\lambda^{s+m+3}$ for some constant $A_{m+s+3}$ we can write
\[ |T_{s,\beta,(\gamma_1,n_1),...,(\gamma_k,n_k)}(x) | \le\frac{c_6 A_{m+s+3}}{3c^mm!} \int_{y+\delta^*}^{\infty} \left |\frac{\partial^{m} D^\beta f}{\partial x_n^{m} }(\overline x,u)\right | \frac{(\delta^*)^2}{(u-y)^3} du.\]
We observe that we are now in the same situation as in the first inequality of \eqref{bullet2} of the case $l=0$ and the same computations lead us to
\[\left( \frac{1}{\phi(r/2)}\int_Q  \left| T_{s,\beta,(\gamma_1,n_1),...,(\gamma_k,n_k)}(x)\right|^p dx\right )^{\frac{1}{p}} \le c_7 \bigl\| \frac{\partial^{m} D\beta f}{\partial x_n^{m} } \bigr\|_{M_p^{\phi,\delta/2}(\Omega)} \] 
for every $Q$ in case 2  and
\[\left( \frac{1}{\phi(r/2)}\int_{Q\cap \Omega^-}  \left| T_{s,\beta,(\gamma_1,n_1),...,(\gamma_k,n_k)}(x)\right|^p dx\right )^{\frac{1}{p}} \le c_8 \bigl\| \frac{\partial^{m} D^\beta f}{\partial x_n^{m} } \bigr\|_{M_p^{\phi,\delta/2}(\Omega)} \] 
for every $Q$ in case 3, where $c_7,c_8$ depend only on $n,l$ and $M$. This concludes also the proof of the case $l>0.$
\end{proof} 

\begin{theorem}\label{speciallip}
Let $1\le p<\infty,n\ge2$, $\phi$ a function from $\rr^+$ to $\rr^+$ and $\Omega$ be a special Lipschitz domain of $\rr^n$ with Lipschitz bound $M.$ Moreover let $S$ be the Stein extension operator. Then for every $f \in W^{l,p}(\Omega)$, every $\delta>0$, and every $\alpha \in \nn_0^n$ with $|\alpha|\le l$ 
\begin{equation}
 \| D^\alpha_w Sf\|_{M_p^{\phi,\delta}(\rr^n)}\le C_{l,n}(M)\sum_{|\beta|\le |\alpha|}\|D^\beta_w f \|_{M_p^{\phi,\delta}(\Omega)} \label{Sbound}
 \end{equation}
 where $C_{l,n}(\Omega)$ depends only on $n,l$ and $M.$

 \begin{proof}
We recall definition of the operator $S.$ Set  $\Gamma$ to be the cone $\Gamma=\{(\overline x, y) \in \rr^n \ | \ M |\overline x|<|y|, y<0 \}$ and let $\eta \in C^\infty_c(\rr^n)$ be a function with total integral 1 and support is contained in $\Gamma.$ Then, given $f \in W^{l,p}(\Omega)$, $Sf$ is defined to be the limit in $W^{l,p}(\rr^n)$ of $Tf_\varepsilon$ as $\varepsilon \to 0,$ where $f_\varepsilon(x)=1/\varepsilon^n \int_{\rr^n} f(x-y)\eta(y/\varepsilon)$ for every $x$ in an appropriate neighborhood of $	\overline \Omega$. We claim that for every $f \in W^{l,p}(\Omega)$, $\delta>0$ and $|\alpha|\le l$
\begin{equation}
\| D^\alpha_w f_\varepsilon\|_{M^{\phi,\delta}_p(\Omega)} \le \| D^\alpha_w f\|_{M^{\phi,\delta}_p(\Omega)} \label{epsbound}.
\end{equation}
To see this first we notice that $D^\alpha_w f_\varepsilon(x)=1/\varepsilon^n \int_{\rr^n} D^\alpha_w f(x-y)\eta(y/\varepsilon)dy$ for every $x \in \Omega.$ Let now $B_{x_0}(r)$ a ball centered in $\Omega$ of radius $0<r<\delta$. By Minkowski's integral inequality
\begin{align*}
 \left( \frac{1}{	\phi(r)} \int_{B_r(x_0)\cap \Omega} |D^\alpha f_\varepsilon (x)|^p \right)^{\frac{1}{p}} &= \left( \frac{1}{	\phi(r)} \int_{B_r(x_0)\cap \Omega} \left |\frac{1}{\varepsilon^n} \int_{\rr^n} D^\alpha_w f(x-y)\eta\left(\frac{y}{\varepsilon} \right)dy\right|^pdx \right)^{\frac{1}{p}} \\
 &\le \frac{1}{\varepsilon^n}  \int_{\rr^n} \eta\left(\frac{y}{\varepsilon} \right) \left (\frac{1}{	\phi(r)}  \int_{B_r(x_0)\cap \Omega} |D^\alpha f(x-y)|^p dx\right)^{\frac{1}{p}}dy \\
 &\le\frac{1}{\varepsilon^n}  \int_{\rr^n} \eta\left(\frac{y}{\varepsilon} \right)  \left (\frac{1}{	\phi(r)}  \int_{B_r(x_0-y)\cap \Omega} |D^\alpha f(x)|^p dx\right)^{\frac{1}{p}}dy \\
 & \le \frac{1}{\varepsilon^n}  \int_{\rr^n} \eta\left(\frac{y}{\varepsilon} \right) \| D^\alpha f\|_{M_p^{\phi,\delta}(\Omega)} dy=\| D^\alpha f\|_{M_p^{\phi,\delta}(\Omega)}
\end{align*}
because $B_r(x_0)\cap \Omega -y \subset B_r(x_0-y)\cap \Omega$ and $x_0-y \in \Omega$ for every $x_0 \in \Omega$ and $y \in \Gamma.$ This proves \eqref{epsbound}. Now combining \eqref{epsbound} with \eqref{Tbound2} we get 
\[ \| D^\alpha Tf_\varepsilon\|_{M_p^{\phi,\delta}(\rr^n)}\le C_{l,n}(M)\sum_{|\beta|\le |\alpha|}\|D^\beta f \|_{M_p^{\phi,\delta}(\Omega)}, \]
for every $\varepsilon>0$ and every $|\alpha|\le l$, with $C_{l,n}(M)$ independent of $\varepsilon$. In particular, for every ball $B$ in $\rr^n$ of radius $\delta>r>0$ we have
\begin{equation}
\left( \frac{1}{\phi(r)}\int_B |D^\alpha Tf_\varepsilon(x)|^p dx \right)^\frac{1}{p} \le C_{l,n}(M)\sum_{|\beta|\le |\alpha|}\|D^\beta f \|_{M_p^{\phi,\delta}(\Omega)} \label{balleps}
\end{equation}
Since $ Tf_\varepsilon$ converges to $Sf$ in $W^{l,p}(\rr^n)$, then  $D^\alpha Tf_\varepsilon$ converges to $D^\alpha_w Sf$ in $L^p(\rr^n)$ for every $|\alpha|\le l$ and as a consequence also in $L^p(B)$ for every ball $B$. Hence we can pass to the limit as $\varepsilon \to 0$ in \eqref{balleps} and obtain
\[\left( \frac{1}{\phi(r)}\int_B |D^\alpha_w S(x)|^p dx \right)^\frac{1}{p} \le C_{l,n}(M)\sum_{|\beta|\le |\alpha|}\|D^\beta_w f \|_{M_p^{\phi,\delta}(\Omega)} \]
for every ball $B$ of radius $r.$ This concludes the proof.
 \end{proof} 
\end{theorem}

\begin{remark}
Theorem \ref{speciallip} holds also if $\Omega$ is a rotation of some Lipschitz domain. This can be shown using Remark \ref{rotlip} and similar computations.
\end{remark}
In Theorem \ref{speciallip} we proved that the Stein operator $S$ preserves the Sobolev-Morrey spaces, in the case of a special Lipschitz domains. Our next goal is to extend this property to the more general Stein operator $E$, defined in \eqref{defEE}, which acts on open set with a minimally smooth boundary. We recall that the Stein operator $E$ for a set $\Omega$ with minimally smooth boundary is defined using a covering $\{ U_i\}_{i=1}^s$ of $\partial \Omega$. The main obstacle to study operator $E$ is that in general there is no regularity conditions for the open sets $U_i$. For this reason we will consider the case when $E$ is constructed with a covering that satisfies some additional hypothesis. To this purpose we define now the notion of special covering for a set with minimally sooth boundary.

\begin{definition}
Let $V$ be an open set in $\rr^n$ and $\varepsilon >0$. We say that $V$  has the $\varepsilon$-ball property if for every $ x\in V $ exists an open ball $B$ of radius $\varepsilon$ contained in $V$ such that $x \in B.$
\end{definition}
 Let $\Omega$ be an open set in $\rr^n$ with minimally smooth boundary with parameters $\varepsilon,M,N$ and a covering $\{ U_i\}_{i=1}^s.$ We say that $\{ U_i\}_{i=1}^s$ is a \textit{special covering} for $\Omega$ if $U_i$ has the $\varepsilon$-ball property for every $i=1,...,s.$ The following proposition shows that such covering exists for every set with with minimally smooth boundary.
 \begin{prop}
 Every open set in $\rr^n $with minimally smooth boundary admits a special covering.
 \end{prop}
 \begin{proof}
 Let $\Omega$ be an open set in $\rr^n$ with minimally smooth boundary with parameters $\varepsilon,M,N$ and a covering $\{ U_i\}_{i=1}^s.$ Let's define
\[V_i:=\bigcup_{\substack{x \in \partial\Omega, \\ B_\varepsilon(x) \subset U_i}}B_\varepsilon(x) \]
and consider the family $\{ V_i\}_{i=1}^{\widetilde s}$ containing the sets $V_i$ that are non-empty. Clearly $V_i$ has the $\varepsilon$-ball property for every $i=1,...,\widetilde s$, hence we just need to show that $\{ V_i\}_{i=1}^{\widetilde s}$ satisfies conditions i),ii),iii) and iv) of Definition \ref{minsmooth} for $\Omega$, with the same constants $\varepsilon,M,N$. To see i) we notice that if $x \in \partial \Omega$ then $B_\varepsilon(x) \subset U_{\overline i}$ for some $\overline i$ and consequently $B_\varepsilon(x) \subset V_{\overline i}.$ ii) follows from the fact that $V_i \subset U_i$ for every $i=1,...,\overline s$. We observe now that for every $i=1,...,\overline s$ there exists a special Lipschitz domain $D_i$ and a rotation $R_i$ of $\rr^n$ such that $U_i\cap \Omega=U_i \cap R_i(D_i).$ Hence $ V_i \cap ( U_i\cap \Omega)=V_i\cap (U_i \cap R_i(D_i))$ and since $V_i \subset U_i$ we obtain
\[ V_i \cap  \Omega=V_i\cap  R_i(D_i).\] This proves both iii) and iv).
\end{proof}


\begin{theorem}

Let $1\le p<\infty,n\ge2$ and $\Omega$ be an open set in $\rr^n$ with minimally smooth boundary. Let $\{ U_i\}_{i=1}^s$ be a special covering for $\Omega.$ Moreover let $E$ be the operator defined in \eqref{defEE} using the sequence $\{ U_i\}_{i=1}^s$. Then if $\Omega$ is bounded, for every $f \in W^{l,p}(\Omega)$, every $\delta>0$ and every $\alpha \in \nn_0^n$ with $|\alpha|\le l$ 
\begin{equation}
 \| D^\alpha_w Ef\|_{M_p^{\phi,\delta}(\rr^n)}\le C\sum_{|\beta|\le |\alpha|}\|D^\beta_w f \|_{M_p^{\phi,\delta}(\Omega)} \label{Ebound1}
 \end{equation}
 where $C$ is independent of $f$ and $\delta.$ If instead $\Omega$ is unbounded, for every $f \in W^{l,p}(\Omega)$ and $\delta>0$
\begin{equation}
 \| D^\alpha_w Ef\|_{M_p^{\phi,\delta}(\rr^n)}\le C_\delta\sum_{|\beta|\le |\alpha|}\|D^\beta_w f \|_{M_p^{\phi,\delta}(\Omega)} \label{Ebound2}
 \end{equation}
 where $C_\delta$ depends on $\delta$ but not on $f.$
\end{theorem}

\begin{proof}
Let  $\varepsilon,N,M$ be the parameters relative to the covering $\{ U_i\}_{i=1}^s$ for $\Omega$. Let $B$ an open ball of radius $0<r<\delta$ in $\rr^n$ and consider the set $J=\{i \in \{1,...,s\} \ | \ B\cap U_i \neq \emptyset\}$. We will prove that $\#J\le c$, where $c$ is a constant that depends only on $\varepsilon,N,\delta,n.$ We consider first the case when $\Omega$ is bounded. Then also its $\varepsilon$-neighborhood $\Omega^\varepsilon $ is bounded. Moreover, by definition $U_i\cap \Omega^\varepsilon$ contains a ball of radius $\varepsilon$, hence $|U_i\cap \Omega^\varepsilon|>\varepsilon^2\omega_n,$ where $\omega_n$ is the volume of the n-dimensional unit ball. Since the covering $\{U_i\}_{i=1}^s$ has multiplicity less than $N$ and $U_i\subset\Omega^\varepsilon$, we have that $\sum_{i=1}^s|U_i\cap \Omega^\varepsilon|\le N |\Omega^\varepsilon|$. This implies that $s\le N |\Omega^\varepsilon|/(\varepsilon^2\omega_n)$ and so $\#J\le N |\Omega^\varepsilon|/(\varepsilon^2\omega_n)=c.$ We observe that in this case $c$ doesn't depend on $\delta.$ Suppose now that $\Omega$ is unbounded. Since the diameter of $B$ is less than $2\delta$, by Lemma \ref{covering} there exists a family of $m$ balls of radius $\varepsilon$ that covers $B$, where $m$ depends only on $\delta,\varepsilon$ and $n$. Suppose now that $\#J>mp$, for some integer $p\in \nn$, then at least one of these balls intersects at least $p+1$  $U_i$'s. Let's call this ball $B_\varepsilon.$ We know that there exists points $x_i$, $i=1,...,p+1$, with $x_i \in B_\varepsilon \cap U_i.$ Since each $U_i$ has the $\varepsilon$-ball property, there are $B_i$, $i=1,...,p+1$, open balls of radius $\varepsilon$ with $B_i \subset U_i$ and $x_i \in B_i.$ We now label $c_i$ the center of the ball $B_i$ and we notice that the set $\{c_1,...,c_{p+1}\}$ is contained in a ball of radius $2\varepsilon.$ Indeed $|x_i-c_i|\le \varepsilon$ and $x_i \in B_\varepsilon,$ for every $i.$ Therefore by Lemma \ref{covering} we can cover the set $\{c_1,...,c_{p+1}\}$ with $q$ open balls of radius $\varepsilon/2$, where $q$ depends only on $n.$ Now suppose that $p>qN$, then at least one of these balls, that we label $B_{\varepsilon/2}$, contains at least $N+1$ points of $\{c_1,...,c_{p+1}\}.$ Without loss of generality we can suppose that they are $c_1,...,c_{N+1}$, but then we must have that $B_1\cap B_2 \cap ... \cap B_{N+1}\neq \emptyset$. Indeed each of these balls contains the center of $B_{\varepsilon/2}.$ However, since $B_i \subset U_i$ this is in contrast with property ii) of Definition \ref{minsmooth}. Hence we proved that if $\#J\ge mp$ then $p\le qN,$ hence $\#J<m(Np+1) $. This is what we wanted to prove. Now that we proved this estimate we can proceed with the proof of the theorem in the case $|\alpha|=0.$ Let $f \in W^{l,p}(\Omega)$, by applying the definition of $Ef$ we get

\begin{align*}
&\left( \frac{1}{\phi(r)} \int_B |Ef(x)|^pdx \right)^\frac{1}{p} \\
&\le\left( \frac{1}{\phi(r)} \int_B \left|\Lambda_+(x) \frac{\sum_{i=1}^s \lambda_i(x)S_i(f\lambda_i)(x)}{\sum_{i=1}^s \lambda_i^2(x)}\right|^pdx \right)^\frac{1}{p}+\left( \frac{1}{\phi(r)} \int_B |\Lambda_-(x)f(x) |^pdx \right)^\frac{1}{p}.
\end{align*}
The second integral can be bound as follows
\begin{align*}
\left( \frac{1}{\phi(r)} \int_B |\Lambda_-(x)f(x) |^pdx \right)^\frac{1}{p}&\le\left( \frac{1}{\phi(r)} \int_{B\cap\Omega} |f(x) |^pdx \right)^\frac{1}{p} \\
& \le \sum_{j=1}^m \left( \frac{1}{\phi(r)} \int_{B_j\cap\Omega} |f(x) |^pdx \right)^\frac{1}{p}\le m\|f\|_{M_p^{\phi,\delta}(\Omega)} \addtag \label{ezbound}
\end{align*}
where $B_1,...,B_m$ is a collection of balls of radius $r<\delta$ centered in $\Omega$ with $m$ depending only on $n.$ To bound the first integral we will use that $\sum_{i=1}^s \lambda_i^2(x)\ge 1$ whenever $x \in \supp \Lambda_+$ and that $\supp \lambda_i \subset U_i$. Moreover we recall that exist rigid rotations $R_i$ and special Lipschitz domains $D_i$ such that $U_i\cap \Omega=U_i\cap R_i(D_i)$. We have

\begin{align*}
&\left( \frac{1}{\phi(r)} \int_B \left|\Lambda_+(x) \frac{\sum_{i=1}^s \lambda_i(x)S_i(f\lambda_i)(x)}{\sum_{i=1}^s \lambda_i^2(x)}\right|^pdx \right)^\frac{1}{p}\le   \left( \frac{1}{\phi(r)} \int_{B}|\sum_{i=1}^s \lambda_i(x)S_i(f\lambda_i)(x)|^pdx \right)^\frac{1}{p} \\
& \le  \sum_{i\in J}  \left( \frac{1}{\phi(r)} \int_{B} |S_i(f\lambda_i)(x)|^pdx \right)^\frac{1}{p} \le \sum_{i\in J} \| S_i(f\lambda_i)\|_{M_p^{\phi,\delta}(\rr^n)} \\
& \le C_n(M) \sum_{i\in J} \| f\lambda_i\|_{M_p^{\phi,\delta}(R_i(D_i))} \le C_n(M) \sum_{i\in J} \| f\|_{M_p^{\phi,\delta}(R_i(D_i)\cap U_i)}=\\
&= C_n(M)\sum_{i\in J} \| f\|_{M_p^{\phi,\delta}(\Omega \cap U_i)} \le C_n(M) c\| f\|_{M_p^{\phi,\delta}(\Omega )}.
\end{align*}
Here we have used inequality \eqref{Sbound} for $S_i$ and $C_n(M)$ is a constant depending only on $n$ and $M$. This combined with \eqref{ezbound} proves \eqref{Ebound2} when $|\alpha|=0.$ We prove now \eqref{Ebound2} when $|\alpha|>0.$ Let's first define the functions
\[ \mu_i=\frac{\Lambda_+\lambda_i}{\sum_{j=1}^s\lambda^2_j}\]
for every $i=1,...,s.$ Then we can rewrite $Ef$ as
\[ Ef(x)=\sum_{i=1}^s \mu_i(x)S_i(f\lambda_i)(x)+\Lambda_-(x)f(x).\]
We recall that every $\lambda_i$ has all bounded derivatives with a bound independent of $i$ and that $\sum_{j=1}^s\lambda^2_j(x)\ge 1$ when $x \in \supp \Lambda_+.$ Moreover for every $x\in \rr^n$ the sum $\sum_{i=1}^s\lambda_i(x)$ has at most $N$ terms different from $0$. Using these facts and the Leibeniz rule it can be proved that also every $\mu_i$ has all bounded derivatives with a bound independent of $i$. Let's consider again an open ball $B$ in $\rr^n$ of radius $r<\delta$ and the set $J=\{i \in \{1,...,s\} \ | \ B\cap U_i \neq \emptyset\}$. For every $x \in B$ we have
\[ Ef(x)=\sum_{i \in J} \mu_i(x)S_i(f\lambda_i)(x)+\Lambda_-(x)f(x).\]
and since the set $J$ is finite we deduce
\[ D^\alpha_wEf(x)=\sum_{i \in J} D^\alpha_w(\mu_i(x)S_i(f\lambda_i)(x))+D^\alpha_w(\Lambda_-(x)f(x)).\]
Now using the Leibeniz rule we get
\[ |D^\alpha_wEf(x)|\le C_\alpha\sum_{i \in J} \sum_{\beta\le \alpha} |D^\beta_wS_i(f\lambda_i)(x)|+C_\alpha\sum_{\beta\le \alpha} |D^\beta_w f(x)|\mathbbm{1}_\Omega(x)\]
where $C_\alpha$ is a constant depending only on $\alpha,n$ and on the bound of the derivatives of $\mu_i$ from order 0 up to order $|\alpha|$, but independent of $i.$ Hence 
\begin{align*}
&\left( \frac{1}{\phi(r)} \int_B |D^\alpha_w Ef(x)|^pdx \right)^\frac{1}{p} \\
&\le C_\alpha\sum_{i\in J}\sum_{\beta\le \alpha}  \left( \frac{1}{\phi(r)} \int_B |D^\beta_wS_i(f\lambda_i)(x)|^pdx \right)^\frac{1}{p} +C_\alpha \sum_{\beta\le \alpha}  \left( \frac{1}{\phi(r)} \int_{B\cap\Omega} |D^\beta_w f(x)|^pdx \right)^\frac{1}{p}.
\end{align*}
Arguing as before we can estimate the second integral as follows
\begin{equation}
C_\alpha \sum_{\beta\le \alpha}  \left( \frac{1}{\phi(r)} \int_{B\cap\Omega} |D^\beta_w f(x)|^pdx \right)^\frac{1}{p}\le  C_\alpha m\sum_{\beta\le \alpha}  \| D^\beta wf\|_{M_p^{\phi,\delta}(\Omega)}. \label{ezbound2}
\end{equation}
We can estimate the first integral using inequality \eqref{Sbound} for $S_i$. In particular we get
\begin{align*}
&C_\alpha\sum_{i\in J}\sum_{\beta\le \alpha}  \left( \frac{1}{\phi(r)} \int_B |D^\beta_wS_i(f\lambda_i)(x)|^pdx \right)^\frac{1}{p} \\
& \le C_{l,n}(M) C_\alpha\sum_{i\in J}\sum_{\beta\le \alpha}\sum_{|\gamma|\le |\beta|} \| D^\gamma_w(\lambda_if)\|_{M_p^{\phi,\delta}(R_i(D_i))} \\
&\le C_\alpha C_{l,n}(M) D\sum_{i\in J}\sum_{\beta\le \alpha}\sum_{|\gamma|\le |\beta|} \| D^\gamma_wf\|_{M_p^{\phi,\delta}(R_i(D_i)\cap U_i)}= \\
&= C_{l,n}(M)C_\alpha D\sum_{i\in J}\sum_{\beta\le \alpha}\sum_{|\gamma|\le |\beta|} \| D^\gamma_wf\|_{M_p^{\phi,\delta}(\Omega\cap U_i)} \\
&\le C_{l,n}(M)C_\alpha m \widetilde D\sum_{i\in J} \sum_{\beta\le \alpha}\| D^\beta_wf\|_{M_p^{\phi,\delta}(\Omega)} \\
&\le C_{l,n}(M) C_\alpha m \widetilde Dc\sum_{\beta\le \alpha}\| D^\beta_wf\|_{M_p^{\phi,\delta}(\Omega)}, \addtag \label{hardbound}
\end{align*}
where $D,\widetilde D$ are constants depending only on $n$ and the bound on the derivatives of $\lambda_i.$ Inequality \eqref{hardbound} together with \eqref{ezbound2} gives \eqref{Ebound2} for $|\alpha|>0.$ We finally observe that in the proof of \eqref{Ebound2} the only constant depending on $\delta$ is $c$, but we know that if $\Omega$ is bounded, $c$ doesn't actually depend on $\delta$. This proves \eqref{Ebound1}.

\end{proof}
\bibliographystyle{plain}
\bibliography{biblio}



\end{document}